%!TEX encoding = UTF-8 Unicode

\documentclass{article}

%% ODER: format ==         = "\mathrel{==}"
%% ODER: format /=         = "\neq "
%
%
\makeatletter
\@ifundefined{lhs2tex.lhs2tex.sty.read}%
  {\@namedef{lhs2tex.lhs2tex.sty.read}{}%
   \newcommand\SkipToFmtEnd{}%
   \newcommand\EndFmtInput{}%
   \long\def\SkipToFmtEnd#1\EndFmtInput{}%
  }\SkipToFmtEnd

\newcommand\ReadOnlyOnce[1]{\@ifundefined{#1}{\@namedef{#1}{}}\SkipToFmtEnd}
\usepackage{amstext}
\usepackage{amssymb}
\usepackage{stmaryrd}
\DeclareFontFamily{OT1}{cmtex}{}
\DeclareFontShape{OT1}{cmtex}{m}{n}
  {<5><6><7><8>cmtex8
   <9>cmtex9
   <10><10.95><12><14.4><17.28><20.74><24.88>cmtex10}{}
\DeclareFontShape{OT1}{cmtex}{m}{it}
  {<-> ssub * cmtt/m/it}{}
\newcommand{\texfamily}{\fontfamily{cmtex}\selectfont}
\DeclareFontShape{OT1}{cmtt}{bx}{n}
  {<5><6><7><8>cmtt8
   <9>cmbtt9
   <10><10.95><12><14.4><17.28><20.74><24.88>cmbtt10}{}
\DeclareFontShape{OT1}{cmtex}{bx}{n}
  {<-> ssub * cmtt/bx/n}{}
\newcommand{\tex}[1]{\text{\texfamily#1}}	% NEU

\newcommand{\Sp}{\hskip.33334em\relax}


\newcommand{\Conid}[1]{\mathit{#1}}
\newcommand{\Varid}[1]{\mathit{#1}}
\newcommand{\anonymous}{\kern0.06em \vbox{\hrule\@width.5em}}
\newcommand{\plus}{\mathbin{+\!\!\!+}}
\newcommand{\bind}{\mathbin{>\!\!\!>\mkern-6.7mu=}}
\newcommand{\rbind}{\mathbin{=\mkern-6.7mu<\!\!\!<}}% suggested by Neil Mitchell
\newcommand{\sequ}{\mathbin{>\!\!\!>}}
\renewcommand{\leq}{\leqslant}
\renewcommand{\geq}{\geqslant}
\usepackage{polytable}

%mathindent has to be defined
\@ifundefined{mathindent}%
  {\newdimen\mathindent\mathindent\leftmargini}%
  {}%

\def\resethooks{%
  \global\let\SaveRestoreHook\empty
  \global\let\ColumnHook\empty}
\newcommand*{\savecolumns}[1][default]%
  {\g@addto@macro\SaveRestoreHook{\savecolumns[#1]}}
\newcommand*{\restorecolumns}[1][default]%
  {\g@addto@macro\SaveRestoreHook{\restorecolumns[#1]}}
\newcommand*{\aligncolumn}[2]%
  {\g@addto@macro\ColumnHook{\column{#1}{#2}}}

\resethooks

\newcommand{\onelinecommentchars}{\quad-{}- }
\newcommand{\commentbeginchars}{\enskip\{-}
\newcommand{\commentendchars}{-\}\enskip}

\newcommand{\visiblecomments}{%
  \let\onelinecomment=\onelinecommentchars
  \let\commentbegin=\commentbeginchars
  \let\commentend=\commentendchars}

\newcommand{\invisiblecomments}{%
  \let\onelinecomment=\empty
  \let\commentbegin=\empty
  \let\commentend=\empty}

\visiblecomments

\newlength{\blanklineskip}
\setlength{\blanklineskip}{0.66084ex}

\newcommand{\hsindent}[1]{\quad}% default is fixed indentation
\let\hspre\empty
\let\hspost\empty
\newcommand{\NB}{\textbf{NB}}
\newcommand{\Todo}[1]{$\langle$\textbf{To do:}~#1$\rangle$}

\EndFmtInput
\makeatother
%
%
%
%
%
%
% This package provides two environments suitable to take the place
% of hscode, called "plainhscode" and "arrayhscode". 
%
% The plain environment surrounds each code block by vertical space,
% and it uses \abovedisplayskip and \belowdisplayskip to get spacing
% similar to formulas. Note that if these dimensions are changed,
% the spacing around displayed math formulas changes as well.
% All code is indented using \leftskip.
%
% Changed 19.08.2004 to reflect changes in colorcode. Should work with
% CodeGroup.sty.
%
\ReadOnlyOnce{polycode.fmt}%
\makeatletter

\newcommand{\hsnewpar}[1]%
  {{\parskip=0pt\parindent=0pt\par\vskip #1\noindent}}

% can be used, for instance, to redefine the code size, by setting the
% command to \small or something alike
\newcommand{\hscodestyle}{}

% The command \sethscode can be used to switch the code formatting
% behaviour by mapping the hscode environment in the subst directive
% to a new LaTeX environment.

\newcommand{\sethscode}[1]%
  {\expandafter\let\expandafter\hscode\csname #1\endcsname
   \expandafter\let\expandafter\endhscode\csname end#1\endcsname}

% "compatibility" mode restores the non-polycode.fmt layout.

\newenvironment{compathscode}%
  {\par\noindent
   \advance\leftskip\mathindent
   \hscodestyle
   \let\\=\@normalcr
   \let\hspre\(\let\hspost\)%
   \pboxed}%
  {\endpboxed\)%
   \par\noindent
   \ignorespacesafterend}

\newcommand{\compaths}{\sethscode{compathscode}}

% "plain" mode is the proposed default.
% It should now work with \centering.
% This required some changes. The old version
% is still available for reference as oldplainhscode.

\newenvironment{plainhscode}%
  {\hsnewpar\abovedisplayskip
   \advance\leftskip\mathindent
   \hscodestyle
   \let\hspre\(\let\hspost\)%
   \pboxed}%
  {\endpboxed%
   \hsnewpar\belowdisplayskip
   \ignorespacesafterend}

\newenvironment{oldplainhscode}%
  {\hsnewpar\abovedisplayskip
   \advance\leftskip\mathindent
   \hscodestyle
   \let\\=\@normalcr
   \(\pboxed}%
  {\endpboxed\)%
   \hsnewpar\belowdisplayskip
   \ignorespacesafterend}

% Here, we make plainhscode the default environment.

\newcommand{\plainhs}{\sethscode{plainhscode}}
\newcommand{\oldplainhs}{\sethscode{oldplainhscode}}
\plainhs

% The arrayhscode is like plain, but makes use of polytable's
% parray environment which disallows page breaks in code blocks.

\newenvironment{arrayhscode}%
  {\hsnewpar\abovedisplayskip
   \advance\leftskip\mathindent
   \hscodestyle
   \let\\=\@normalcr
   \(\parray}%
  {\endparray\)%
   \hsnewpar\belowdisplayskip
   \ignorespacesafterend}

\newcommand{\arrayhs}{\sethscode{arrayhscode}}

% The mathhscode environment also makes use of polytable's parray 
% environment. It is supposed to be used only inside math mode 
% (I used it to typeset the type rules in my thesis).

\newenvironment{mathhscode}%
  {\parray}{\endparray}

\newcommand{\mathhs}{\sethscode{mathhscode}}

% texths is similar to mathhs, but works in text mode.

\newenvironment{texthscode}%
  {\(\parray}{\endparray\)}

\newcommand{\texths}{\sethscode{texthscode}}

% The framed environment places code in a framed box.

\def\codeframewidth{\arrayrulewidth}
\RequirePackage{calc}

\newenvironment{framedhscode}%
  {\parskip=\abovedisplayskip\par\noindent
   \hscodestyle
   \arrayrulewidth=\codeframewidth
   \tabular{@{}|p{\linewidth-2\arraycolsep-2\arrayrulewidth-2pt}|@{}}%
   \hline\framedhslinecorrect\\{-1.5ex}%
   \let\endoflinesave=\\
   \let\\=\@normalcr
   \(\pboxed}%
  {\endpboxed\)%
   \framedhslinecorrect\endoflinesave{.5ex}\hline
   \endtabular
   \parskip=\belowdisplayskip\par\noindent
   \ignorespacesafterend}

\newcommand{\framedhslinecorrect}[2]%
  {#1[#2]}

\newcommand{\framedhs}{\sethscode{framedhscode}}

% The inlinehscode environment is an experimental environment
% that can be used to typeset displayed code inline.

\newenvironment{inlinehscode}%
  {\(\def\column##1##2{}%
   \let\>\undefined\let\<\undefined\let\\\undefined
   \newcommand\>[1][]{}\newcommand\<[1][]{}\newcommand\\[1][]{}%
   \def\fromto##1##2##3{##3}%
   \def\nextline{}}{\) }%

\newcommand{\inlinehs}{\sethscode{inlinehscode}}

% The joincode environment is a separate environment that
% can be used to surround and thereby connect multiple code
% blocks.

\newenvironment{joincode}%
  {\let\orighscode=\hscode
   \let\origendhscode=\endhscode
   \def\endhscode{\def\hscode{\endgroup\def\@currenvir{hscode}\\}\begingroup}
   %\let\SaveRestoreHook=\empty
   %\let\ColumnHook=\empty
   %\let\resethooks=\empty
   \orighscode\def\hscode{\endgroup\def\@currenvir{hscode}}}%
  {\origendhscode
   \global\let\hscode=\orighscode
   \global\let\endhscode=\origendhscode}%

\makeatother
\EndFmtInput
%
%


\usepackage{xcolor}

% Char literal
\colorlet{Char}{darkgray}

% Numeral literal
\colorlet{Numeral}{Char}

% Keyword
\colorlet{Keyword}{red!50!black}

% Module identifier
\colorlet{ModId}{Char}

% Variable identifier and symbol
\colorlet{VarId}{Char}
\colorlet{VarSym}{VarId}

% Data constructor identifier and symbol
\colorlet{ConId}{VarId}
\colorlet{ConSym}{ConId}

% Type variable identifier
\colorlet{TVarId}{gray}

% Type constructor identifier
\colorlet{TConId}{TVarId}
\colorlet{TConSym}{TConId}

% Type class identifier
\colorlet{TClassId}{TVarId}

% Comment
\colorlet{Comment}{blue!45!black}

\newcommand\Char[1]{\textcolor{Char}{\texttt{#1}}}

% Numeral literal
\newcommand\Numeral[1]{\textcolor{Numeral}{\mathsf{#1}}}

% Keyword
\newcommand\Keyword[1]{\textcolor{Keyword}{\textbf{\textsf{#1}}}}

% Module identifier
\newcommand\ModId[1]{\mathord{\textcolor{ModId}{\mathsf{#1}}}}

% Variable identifier and symbol
\newcommand\VarId[1]{\mathord{\textcolor{VarId}{#1}}}
\let\Varid\VarId
\newcommand\VarSym[1]{\mathbin{\textcolor{VarSym}{#1}}}

% Data constructor identifier and symbol
\newcommand\ConId[1]{\mathord{\textcolor{ConId}{\mathsf{#1}}}}
\let\Conid\ConId
\newcommand\ConSym[1]{\mathbin{\textcolor{ConSym}{\mathsf{#1}}}}

% Type variable identifier
\newcommand\TVarId[1]{\mathord{\textcolor{TVarId}{\mathsf{#1}}}}

% Type constructor identifier
\newcommand\TConId[1]{\mathord{\textcolor{TConId}{\mathsf{#1}}}}
% \newcommand\TConSym[1]{\mathbin{#1}}
\newcommand\TConSym[1]{\mathbin{\textcolor{TConSym}{\mathsf{#1}}}}

% Type class identifier
\newcommand\TClassId[1]{\mathord{\textcolor{TClassId}{\textit{\textsf{#1}}}}}

% Comment
\newcommand\Comment[1]{\textcolor{Comment}{\textit{\textsf{#1}}}}

% Package identifier (used in text, not code/math environment)
\newcommand\PkgId[1]{\textcolor{Char}{\texttt{#1}}}




% Comments: one-line, nested, pragmas



\DeclareMathAlphabet{\mathkw}{OT1}{cmss}{bx}{n}

\newcommand{\FN}{\mathsf}
\newcommand{\comment}[1]{\marginpar{#1}}
\newcommand{\ignore}[1]{}




















\usepackage{setspace,homework,stmaryrd,wasysym,url,upgreek,subfigure}
\usepackage[margin=1cm]{caption}
\usepackage{xytree, listings, linguex, qtree}
\usepackage[toc,page]{appendix}

\definecolor{gray}{rgb}{0.4,0.4,0.4}

\lstset{
  basicstyle=\ttfamily,
  columns=fullflexible,
  showstringspaces=false,
  commentstyle=\color{gray}\upshape
}

\lstdefinelanguage{XML}
{
  morestring=[b]",
  morestring=[s]{>}{<},
  morecomment=[s]{<?}{?>},
  stringstyle=\color{gray},
  identifierstyle=\color{black},
  keywordstyle=\color{black},
  morekeywords={xmlns,version,type}% list your attributes here
}


\begin{document}
\onehalfspacing
\setmainfont{Times New Roman}

\author{Jonathan Sterling}

\title{A Survey of Phrase Projectivity in \emph{Antigone}}
\date{April 2013}
\maketitle


\noindent
\textbf{THIS IS A DRAFT. It is currently lacking a bibliography, and the
analysis is a bit shallow currently.}

In this paper, I will show how (and to what degree) phrase projectivity
corresponds with register and meter in Sophocles's \emph{Antigone}, by
developing a quantitative metric for projectivity and comparing it across
lyrics, trimeters and anapaests.

\section{Dependency Trees and Their Projectivity}

A dependency tree encodes the head-dependent relation for a string of words,
where arcs are drawn from heads to their dependents. We consider a phrase
\emph{projective} when these arcs do not cross each other, and
\emph{discontinuous} to the extent that any of the arcs intersect.
Figure~\ref{fig:dependency-trees1} is a minimal pair that demonstrates how
hyperbaton introduces a projectivity violation; in this case, a path of
dependency ``wraps around itself''.

\begin{figure}[h!]
\centering
\subfigure[``Full of plentiful supplies'' (Xenophon, \emph{Anabasis} 3.5.1) is fully projective.]{
  \xytext{
    \xybarnode{μεστῇ}\xybarconnect[6](U,U){2}&
    \xybarnode{πολλῶν}&
    \xybarnode{ἀγαθῶν}\xybarconnect[3](UL,U){-1}
  }
}
\hspace{6pt}
\subfigure[``Full of many good things'' (Plato, \emph{Laws} 906a) has one
projectivity violation.]{
  \xytext{
    \xybarnode{πολλῶν}&
    \xybarnode{μεστὸν}\xybarconnect[6]{1}&
    \xybarnode{ἀγαθῶν}\xybarconnect[3](UL,U){-2}
  }
}

\caption{Examples drawn from Devine~\&~Stephens.}
\label{fig:dependency-trees1}
\end{figure}

In addition to the above, adjacent phrases (that is, phrases at the same level
in the tree) may interlace, causing projectivity violations. This is commonly
introduced by Wackernagel's Law, as in Figure~\ref{fig:wackernagel}, where the
placement of μὲν δὴ interlaces with the τὰ...πόλεος phrase.

\begin{figure}[h!]
\centering
\xytext{
    \xybarnode{τὰ}&
    \xybarnode{μὲν}\xybarconnect[3](UR,U){1}&
    \xybarnode{δὴ}&
    \xybarnode{πόλεος}\xybarconnect[6](UL,U){-3}&
    \xybarnode{...}&
    \xybarnode{ὤρθησαν}\xybarconnect[9]{-2}\xybarconnect[9](UR,U){-4}
}
\caption{``[The gods] righted the matters of the city...'' (\emph{Antigone}
162--163) has
one projectivity violation, due to the μὲν δὴ falling in Wackernagel's Position.}
\label{fig:wackernagel}
\end{figure}

We consider a violation to have occured for each and every intersection of lines
on such a drawing; thus, the hyperbaton of one word may introduce multiple
violations. Consider, for instance, Figure~\ref{fig:stas-tree}, in which five
violations are brought about by the displacement of φονώσαισιν. In this way, the
number of intersections is a good heuristic for judging the severity of
hyperbata.

\begin{figure}[h!]
\centering
\xytext{
  \xybarnode{στὰς}\xybarconnect[6]{2} &
  \xybarnode{δ'} &
  \xybarnode{ὑπὲρ}\xybarconnect[3](UR,U){1} &
  \xybarnode{μελάθρων} &
  \xybarnode{φονώσαισιν} &
  \xybarnode{ἀμφι}
    \xybarconnect[6](U,U){3} &
  \xybarnode{\!\!\!χανὼν}
    \xybarconnect[6](U,U){1}
    \xybarconnect[6](UL,U){4}&
  \xybarnode{κύκλῳ} &
  \xybarnode{λόγχαις}\xybarconnect[3](UL,U){-4} &
  \xybarnode{ἑπτάπυλον} &
  \xybarnode{στόμα}\xybarconnect[3](UL,U){-1} &
  \xybarnode{ἔβα}
    \xybarconnect[9](UL,UL){-11}
    \xybarconnect[9](U,U){-10}
    \xybarconnect[9](UR,UL){-6}
}
\caption{``And he stood over the rooftops, gaped in a circle with murderous
spears around the seven-gated mouth, and left'' (\emph{Antigone}
117--120) has six projectivity violations.}
\label{fig:stas-tree}
\end{figure}

\subsection{Counting Violations}

Drawing trees and counting intersections is time-consuming and error-prone,
especially since the number of intersections may vary if one is not consistent
with the relative height of arcs. It is clear, then, that a computer ought to be
able to do the job faster and more accurately than a human, given at least the
head-dependent relations for a corpus.

The formal algorithm for counting the number of intersections is given in
Appendix~\ref{sec:algorithm}, but I shall reproduce an informal and mostly
nontechnical version of it here. First, we index each word in the sentence by
its linear position, and cross-reference it with the linear position of its
head:

\begin{quote}
\gll στὰς δ' ὑπὲρ μελάθρων φονώσαισιν ἀμφικανὼν κύκλῳ λόγχαις ἑπτάπυλον στόμα
ἔβα\\
      1:11  2:11 3:1 4:3 5:8 6:11 7:6 8:6 9:10 10:6 11:\_\\
\end{quote}

\noindent
Next, arrange the dependencies into a tree as in Figure~\ref{fig:rose-tree}.
Then, counting upwards from the lowest edges (i.e.\ the lines) in the tree up to
the topmost ones, make a list of edges indexed by vertical level as in
Table~\ref{tab:edges}.

\begin{figure}
  \Tree
  [.11
    [.1 [.3 4 ] ]
    2
    [.6
      7
      [.8 5 ]
      [.10 9 ]
    ]
  ]
\caption{The dependency relations arranged in a non-linear tree.}
\label{fig:rose-tree}
\end{figure}

\begin{table}
\centering
  \begin{tabular}{cl}
  \toprule
  \emph{level} & \emph{edges}\\
  \midrule
  1 & \ensuremath{\Numeral{3}\leftrightarrow\Numeral{4},\Numeral{5}\leftrightarrow\Numeral{8},\Numeral{9}\leftrightarrow\Numeral{10}}\\
  2 & \ensuremath{\Numeral{1}\leftrightarrow\Numeral{3},\Numeral{6}\leftrightarrow\Numeral{7},\Numeral{6}\leftrightarrow\Numeral{8},\Numeral{6}\leftrightarrow\Numeral{10}}\\
  3 & \ensuremath{\Numeral{1}\leftrightarrow\Numeral{11},\Numeral{2}\leftrightarrow\Numeral{11},\Numeral{6}\leftrightarrow\Numeral{11}}
  \\
  \bottomrule
  \end{tabular}
  \caption{Edges of the tree in Figure~\ref{fig:rose-tree} arranged by level.}
  \label{tab:edges}
\end{table}

Then, each edge in our table must be checked for violations against all the
other edges in the table except those which are in a level higher than it. The
level of the edge corresponds with the height at which we drew the arcs; this
condition arises out of the fact that an arc cannot cross an arc that is above
it, rather, only one that is below it.

Next, we must figure out all the possible ways for an arc to intersect another
at given levels. These are enumerated in detail in the function \ensuremath{\FN{checkEdges}} in
Appendix~\ref{sec:counting}, but suffice it to say that they fall into a few
main categories:

\begin{enumerate}
\item Both vertices of the higher edge are within the bounds of the lower edge.
This is a double violation, as both sides of an arc will extrude through
another.
\item One vertex of the higher edge is within the bounds of the lower edge, and
the other vertex is not; this vertex is allowed to be equal to the second vertex
of the lower edge. In either case, this is a single violation, as just one
intersection occurs.
\item The edges are at the same level, and one vertex of the higher edge is
neither within bounds of the other, nor equal to any of the vertices of the
other.
\end{enumerate}

\noindent
Using this procedure, we shall have found the edge violations which are listed
in Table~\ref{tab:violations}, which are \ensuremath{\Numeral{6}} in total.

\begin{table}
\centering
\begin{tabular}{ccc}
\toprule
\ensuremath{\Numeral{2}\leftrightarrow\Numeral{11}} & \ensuremath{\Numeral{1}\leftrightarrow\Numeral{3}} & 1\\
\ensuremath{\Numeral{6}\leftrightarrow\Numeral{11}} & \ensuremath{\Numeral{5}\leftrightarrow\Numeral{8}} & 1\\
\ensuremath{\Numeral{6}\leftrightarrow\Numeral{7}}  & \ensuremath{\Numeral{5}\leftrightarrow\Numeral{8}} & 2\\
\ensuremath{\Numeral{6}\leftrightarrow\Numeral{8}}  & \ensuremath{\Numeral{5}\leftrightarrow\Numeral{8}} & 1\\
\ensuremath{\Numeral{6}\leftrightarrow\Numeral{10}} & \ensuremath{\Numeral{5}\leftrightarrow\Numeral{8}}  & 1\\
\midrule
\multicolumn{3}{r}{$\FN{total}$ \ensuremath{\mathrel{=}\Numeral{6}}}\\
\bottomrule
\end{tabular}
\caption{Projectivity violations which arise from the edges in
Table~\ref{tab:edges}.}
\label{tab:violations}
\end{table}

\subsection{\ensuremath{\FN{\omega}}: a metric of projectivity}

In order for our view of a text's overall projectivity to not be skewed by its
length, we must have a ratio. For the purposes of this paper, we shall call this
metric $\omega$, as given by the following ratio:
%
\[ \omega = \frac{\text{number of violations}}{\text{number of arcs}} \]
%

\section{The Perseus Treebank}
%
The Perseus Ancient Greek Dependency Treebank is a massive trove of annotated
texts that encode the all dependency relations in every sentence. The data is
given in an XML (E\textbf{x}tensible \textbf{M}arkup \textbf{L}anguage) format
resembling the following:

\lstset{
  language=XML,
  escapeinside=**
}

\begin{lstlisting}
    <sentence id="2900759">
      <word id="1" form="*\color{gray}\textrm{χρὴ}*" lemma="*\color{gray}\textrm{χρή}*" head="0" />
      <word id="2" form="*\color{gray}\textrm{δὲ}*" lemma="*\color{gray}\textrm{δέ}*" head="1" />
      ...
    </sentence>
\end{lstlisting}

\noindent
%
Every sentence is given a unique, sequential identifier; within each sentence,
every word is indexed by its linear position and coreferenced with the linear
position of its dominating head. In the case of the data for \emph{Antigone},
the maximal head of each sentence has its own head given as \lstinline{0}.
Appendix~\ref{sec:parsing} deals with parsing these XML representations into
dependency trees for which we can compute \ensuremath{\FN{\omega}}.


\section{Projectivity in Antigone}

To observe the variation of projectivity within a text, then, one may make a
selection of sentences that have something in common, compute their trees and
thence derive \ensuremath{\FN{\omega}}, and then average the results. Then that quantity may be
compared with that of other selections.

I have chosen to compare projectivity in lyrics, anapaests and trimeters. Lyrics
I have divided into two categories: choral odes and laments, whereas I divide
trimeters into medium-to-long speeches and stichomythia.

To that end, I have selected passages from \emph{Antigone} and organized them by
type. Table~\ref{tab:lyrics} enumerates the lyric passages of the play, along
with their computed mean \ensuremath{\FN{\omega}} values, and a final mean of means with the
standard deviation of the set.  Table~\ref{tab:anapaests} does the same for
anapaests. Lastly, Table~\ref{tab:dialogue} gives selections of dialogue (which
is in iambic trimeters), divided between medium-to-long speeches and
stichomythia.

\begin{table}
  \centering
  \subtable[Choral Odes]{
\begin{tabular}{clc}\toprule\textbf{Lines}&\textbf{}&\ensuremath{\FN{\omega}}\\ \midrule\ensuremath{\Numeral{100}\cdots\Numeral{154}}&\emph{First choral ode}&\ensuremath{\Numeral{0.77}}\\\ensuremath{\Numeral{332}\cdots\Numeral{375}}&\emph{Second choral ode}&\ensuremath{\Numeral{0.47}}\\\ensuremath{\Numeral{583}\cdots\Numeral{625}}&\emph{Third choral ode}&\ensuremath{\Numeral{0.53}}\\\ensuremath{\Numeral{781}\cdots\Numeral{800}}&\emph{Fourth choral ode}&\ensuremath{\Numeral{0.47}}\\\ensuremath{\Numeral{944}\cdots\Numeral{987}}&\emph{Fifth choral ode}&\ensuremath{\Numeral{0.41}}\\\ensuremath{\Numeral{1116}\cdots\Numeral{1152}}&\emph{Sixth choral ode}&\ensuremath{\Numeral{0.75}}\\\midrule\multicolumn{3}{r}{\ensuremath{\FN{mean}\;\FN{\omega}} = \ensuremath{\Numeral{0.57}}, \ensuremath{\FN{sdev}} = \ensuremath{\Numeral{0.15}}}\\\bottomrule\end{tabular}
}
  \vspace{6pt}
  \subtable[Laments]{
\begin{tabular}{clc}\toprule\textbf{Lines}&\textbf{}&\ensuremath{\FN{\omega}}\\ \midrule\ensuremath{\Numeral{806}\cdots\Numeral{816}}&\emph{Antigone's Lament}&\ensuremath{\Numeral{1.29}}\\\ensuremath{\Numeral{823}\cdots\Numeral{833}}&\emph{Antigone's Lament (cntd.)}&\ensuremath{\Numeral{0.79}}\\\ensuremath{\Numeral{839}\cdots\Numeral{882}}&\emph{Antigone's Lament (cntd.)}&\ensuremath{\Numeral{0.54}}\\\ensuremath{\Numeral{1261}\cdots\Numeral{1269}}&\emph{Kreon's Lament}&\ensuremath{\Numeral{0.52}}\\\ensuremath{\Numeral{1283}\cdots\Numeral{1292}}&\emph{Kreon's Lament (cntd.)}&\ensuremath{\Numeral{0.88}}\\\ensuremath{\Numeral{1306}\cdots\Numeral{1311}}&\emph{Kreon's Lament (cntd.)}&\ensuremath{\Numeral{0.33}}\\\ensuremath{\Numeral{1317}\cdots\Numeral{1325}}&\emph{Kreon's Lament (cntd.)}&\ensuremath{\Numeral{1.08}}\\\ensuremath{\Numeral{1239}\cdots\Numeral{1246}}&\emph{Kreon's Lament (cntd.)}&\ensuremath{\Numeral{0.47}}\\\midrule\multicolumn{3}{r}{\ensuremath{\FN{mean}\;\FN{\omega}} = \ensuremath{\Numeral{0.74}}, \ensuremath{\FN{sdev}} = \ensuremath{\Numeral{0.33}}}\\\bottomrule\end{tabular}
}
  \caption{Lyrics}
  \label{tab:lyrics}
\end{table}

\begin{table}
  \centering
  
\begin{tabular}{clc}\toprule\textbf{Lines}&\textbf{}&\ensuremath{\FN{\omega}}\\ \midrule\ensuremath{\Numeral{155}\cdots\Numeral{161}}&\emph{Kreon's Entrance}&\ensuremath{\Numeral{0.33}}\\\ensuremath{\Numeral{376}\cdots\Numeral{383}}&\emph{Antigone's Entrance}&\ensuremath{\Numeral{0.64}}\\\ensuremath{\Numeral{526}\cdots\Numeral{530}}&\emph{Ismene's Entrance}&\ensuremath{\Numeral{0.05}}\\\ensuremath{\Numeral{626}\cdots\Numeral{630}}&\emph{Haimon's Entrance}&\ensuremath{\Numeral{0.56}}\\\ensuremath{\Numeral{801}\cdots\Numeral{805}}&\emph{Antigone's Entrance}&\ensuremath{\Numeral{1.16}}\\\ensuremath{\Numeral{817}\cdots\Numeral{822}}&\emph{Chorus to Antigone}&\ensuremath{\Numeral{0.57}}\\\ensuremath{\Numeral{834}\cdots\Numeral{838}}&\emph{Chorus to Antigone}&\ensuremath{\Numeral{0.03}}\\\ensuremath{\Numeral{929}\cdots\Numeral{943}}&\emph{Chorus, Kreon and Antigone}&\ensuremath{\Numeral{0.27}}\\\ensuremath{\Numeral{1257}\cdots\Numeral{1260}}&\emph{Chorus before Kreon's Kommos}&\ensuremath{\Numeral{0.00}}\\\ensuremath{\Numeral{1347}\cdots\Numeral{1353}}&\emph{Final anapaests of the Chorus}&\ensuremath{\Numeral{0.36}}\\\midrule\multicolumn{3}{r}{\ensuremath{\FN{mean}\;\FN{\omega}} = \ensuremath{\Numeral{0.40}}, \ensuremath{\FN{sdev}} = \ensuremath{\Numeral{0.35}}}\\\bottomrule\end{tabular}

  \caption{Anapaests.}
  \label{tab:anapaests}
\end{table}

\begin{table}
  \centering
  \subfigure[Speeches and Dialogue]{
\begin{tabular}{clc}\toprule\textbf{Lines}&\textbf{}&\ensuremath{\FN{\omega}}\\ \midrule\ensuremath{\Numeral{162}\cdots\Numeral{210}}&\emph{\emph{Kreon:} ἄνδρες, τὰ μὲν δὴ...}&\ensuremath{\Numeral{0.31}}\\\ensuremath{\Numeral{249}\cdots\Numeral{277}}&\emph{\emph{Guard:} οὐκ οἶδ'· ἐκεῖ γὰρ οὔτε...}&\ensuremath{\Numeral{0.42}}\\\ensuremath{\Numeral{280}\cdots\Numeral{314}}&\emph{\emph{Kreon:} παῦσαι, πρὶν ὀργῆς...}&\ensuremath{\Numeral{0.45}}\\\ensuremath{\Numeral{407}\cdots\Numeral{440}}&\emph{\emph{Guard:} τοιοῦτον ἦν τὸ πρᾶγμ'...}&\ensuremath{\Numeral{0.50}}\\\ensuremath{\Numeral{450}\cdots\Numeral{470}}&\emph{\emph{Antigone:} οὐ γάρ τί μοι Ζεὺς...}&\ensuremath{\Numeral{0.43}}\\\ensuremath{\Numeral{473}\cdots\Numeral{495}}&\emph{\emph{Kreon:} ἀλλ' ἴσθι τοι...}&\ensuremath{\Numeral{0.56}}\\\ensuremath{\Numeral{639}\cdots\Numeral{680}}&\emph{\emph{Kreon:} οὕτω γὰρ, ὦ παῖ...}&\ensuremath{\Numeral{0.50}}\\\ensuremath{\Numeral{683}\cdots\Numeral{723}}&\emph{\emph{Haimon:} πἀτερ, θεοὶ φύουσιν...}&\ensuremath{\Numeral{0.43}}\\\ensuremath{\Numeral{891}\cdots\Numeral{928}}&\emph{\emph{Antigone:} ὦ τύμβος, ὦ νυμφεῖον...}&\ensuremath{\Numeral{0.37}}\\\ensuremath{\Numeral{998}\cdots\Numeral{1032}}&\emph{\emph{Teiresias:} γνώσῃ, τέχνης σημεῖα...}&\ensuremath{\Numeral{0.42}}\\\ensuremath{\Numeral{1033}\cdots\Numeral{1047}}&\emph{\emph{Kreon:} ὦ πρέσβυ, πάντες...}&\ensuremath{\Numeral{0.22}}\\\ensuremath{\Numeral{1064}\cdots\Numeral{1090}}&\emph{\emph{Teiresias:} ἀλλ' εὖ γέ τοι...}&\ensuremath{\Numeral{0.77}}\\\ensuremath{\Numeral{1155}\cdots\Numeral{1172}}&\emph{\emph{Messenger:} Κάδμου πάροικοι καὶ...}&\ensuremath{\Numeral{0.40}}\\\ensuremath{\Numeral{1192}\cdots\Numeral{1243}}&\emph{\emph{Messenger:} ἐγώ, φίλη δέσποινα...}&\ensuremath{\Numeral{0.34}}\\\midrule\multicolumn{3}{r}{\ensuremath{\FN{mean}\;\FN{\omega}} = \ensuremath{\Numeral{0.44}}, \ensuremath{\FN{sdev}} = \ensuremath{\Numeral{0.13}}}\\\bottomrule\end{tabular}
}
  \vspace{6pt}
  \subfigure[Stichomythia]{
\begin{tabular}{clc}\toprule\textbf{Lines}&\textbf{}&\ensuremath{\FN{\omega}}\\ \midrule\ensuremath{\Numeral{536}\cdots\Numeral{576}}&\emph{Ismene, Antigone and Kreon}&\ensuremath{\Numeral{0.26}}\\\ensuremath{\Numeral{728}\cdots\Numeral{757}}&\emph{Haimon and Kreon}&\ensuremath{\Numeral{0.33}}\\\ensuremath{\Numeral{991}\cdots\Numeral{997}}&\emph{Kreon and Teiresias}&\ensuremath{\Numeral{0.63}}\\\ensuremath{\Numeral{1047}\cdots\Numeral{1063}}&\emph{Kreon and Teiresias}&\ensuremath{\Numeral{0.10}}\\\ensuremath{\Numeral{1172}\cdots\Numeral{1179}}&\emph{Chorus and Messenger}&\ensuremath{\Numeral{0.21}}\\\midrule\multicolumn{3}{r}{\ensuremath{\FN{mean}\;\FN{\omega}} = \ensuremath{\Numeral{0.31}}, \ensuremath{\FN{sdev}} = \ensuremath{\Numeral{0.20}}}\\\bottomrule\end{tabular}
}
  \caption{Dialogue (Trimeters)}
  \label{tab:dialogue}
\end{table}

As can be seen from the data, lyrics have the highest degree of
non-projectivity, followed by speeches, then anapaests, and then stichomythia.
To try and understand why this is the case, it will be useful to discuss Greek
hyperbaton in my general terms.

Whereas in prose, hyperbaton corresponds to \emph{strong focus}, which ``does
not merely fill a gap in the addressee's knowledge but additionally evokes and
excludes alternatives'' (Devine~\&~Stephens 303), hyperbaton in verse only
entails weak focus, which emphasizes but does not exclude (ibid.\ 107).

As a result, hyperbaton in verse may be used to evoke a kind of elevated style
without incidentally entailing more emphasis and other pragmatic effects than
intended. And so it should not be surprising that lyric passages, which reside
in the most poetic and elevated register present in tragic diction, should have
proved in \emph{Antigone} to have the highest proportion of projectivity
violations.

Within the lyric passages, the laments appear to have consistently higher
\ensuremath{\FN{\omega}}s than the choral odes, which may stem from their being much more emotive
and personal in nature. It should be noted that, whilst the odes are very
tightly centered around the mean, there is a fair degree of
variation in the \ensuremath{\FN{\omega}} for the laments.

Likewise, the anapaests vary so wildly in their \ensuremath{\FN{\omega}}s that it may be difficult
to say very much about them that is relevant to the questions we are
considering.

As for dialog, longer-form speeches are tightly wrapped around their mean, with
stichomythias varying a bit more. Speeches are a somewhat less projective than
the stichmythias, being typically more eloquent and long-winded than their
argumentative, choppy counterparts.

So far, the most surprising thing about the data is the degree to which certain
passages vary in \ensuremath{\FN{\omega}} (or, if you like, the degree to which some passages
\emph{don't} vary in \ensuremath{\FN{\omega}}). The data draw us, then, to the following
conclusions:

\begin{enumerate}
\item Non-projectivity varies within a single metrical type: that is,
though lyric passages are in general less projective than anything else, some
laments reach a degree of non-projectivity that exceeds the most elliptical
odes in \emph{Antigone}. Further, within the iambic trimeters, speeches are less
projective than stichomythias.

\item Certain registers seem to be more conventionalized with respect to frequency
of hyperbaton than others; that is, choral odes and speeches do not vary greatly
amongst themselves, but laments and anapaests do.
\end{enumerate}

\noindent
It would then seem that meter itself is not a primary factor for predicting
incidence of hyperbaton, but rather a secondary one only. That is to say, we
know for a fact that passages in lyric meters have greater \ensuremath{\FN{\omega}} than passages
in other meters. Yet, the variation of \ensuremath{\FN{\omega}} within that very meter indicates
that there is some other factor involved, which very likely has to do with
register along two different dimensions, which is to say, relative dignity and
emotive force.

\newpage
\begin{appendices}
\section{Algorithm \& Data Representation}
\label{sec:algorithm}

Dependency trees are a recursive data structure with a head node, which may have
any number of arcs drawn to further trees (this is called a \emph{rose tree}).
We represent them as a Haskell data-type as follows:
\begin{hscode}\SaveRestoreHook
\column{B}{@{}>{\hspre}l<{\hspost}@{}}%
\column{3}{@{}>{\hspre}l<{\hspost}@{}}%
\column{E}{@{}>{\hspre}l<{\hspost}@{}}%
\>[3]{}\mathkw{data}\;\mathsf{Tree}\;\VarSym{\alpha}\mathrel{=}\VarSym{\alpha}\curvearrowright[\mskip1.5mu \mathsf{Tree}\;\VarSym{\alpha}\mskip1.5mu]{}\<[E]%
\ColumnHook
\end{hscode}\resethooks
This can be read as ``For all types \ensuremath{\VarSym{\alpha}}, a \ensuremath{\mathsf{Tree}} of \ensuremath{\VarSym{\alpha}} is constructed from a
\emph{label} of type \ensuremath{\VarSym{\alpha}} and a \emph{subforest} of \ensuremath{\mathsf{Tree}}s of \ensuremath{\VarSym{\alpha}},'' where brackets are a notation
for lists.

Given a tree, we can extract its root label or its subforest by pattern matching
on its structure as follows:
\begin{hscode}\SaveRestoreHook
\column{B}{@{}>{\hspre}l<{\hspost}@{}}%
\column{3}{@{}>{\hspre}l<{\hspost}@{}}%
\column{E}{@{}>{\hspre}l<{\hspost}@{}}%
\>[3]{}\FN{getLabel}\ConSym{::}\mathsf{Tree}\;\VarSym{\alpha}\to \VarSym{\alpha}{}\<[E]%
\\
\>[3]{}\FN{getLabel}\;(\Varid{l}\curvearrowright\anonymous )\mathrel{=}\Varid{l}{}\<[E]%
\\
\>[3]{}\FN{getForest}\ConSym{::}\mathsf{Tree}\;\VarSym{\alpha}\to [\mskip1.5mu \mathsf{Tree}\;\VarSym{\alpha}\mskip1.5mu]{}\<[E]%
\\
\>[3]{}\FN{getForest}\;(\anonymous \curvearrowright\Varid{ts})\mathrel{=}\Varid{ts}{}\<[E]%
\ColumnHook
\end{hscode}\resethooks
\subsection{From Edges to Trees}
\label{sec:edges-to-trees}

We shall consider each word index to be a \emph{vertex}, and each pair of
vertices to be an \ensuremath{\mathsf{Edge}}, which we shall write as follows:
\begin{hscode}\SaveRestoreHook
\column{B}{@{}>{\hspre}l<{\hspost}@{}}%
\column{3}{@{}>{\hspre}l<{\hspost}@{}}%
\column{E}{@{}>{\hspre}l<{\hspost}@{}}%
\>[3]{}\mathkw{data}\;\mathsf{Edge}\;\VarSym{\alpha}\mathrel{=}\VarSym{\alpha}\leftrightarrow\VarSym{\alpha}\;\mathkw{deriving}\;\mathsf{Eq}{}\<[E]%
\ColumnHook
\end{hscode}\resethooks
\ignore{
\begin{hscode}\SaveRestoreHook
\column{B}{@{}>{\hspre}l<{\hspost}@{}}%
\column{3}{@{}>{\hspre}l<{\hspost}@{}}%
\column{E}{@{}>{\hspre}l<{\hspost}@{}}%
\>[3]{}\mathkw{deriving}\;\mathkw{instance}\;\mathsf{Show}\;\VarSym{\alpha}\Rightarrow \mathsf{Show}\;(\mathsf{Edge}\;\VarSym{\alpha}){}\<[E]%
\\
\>[3]{}\mathkw{deriving}\;\mathkw{instance}\;\mathsf{Show}\;\mathsf{Sentence}{}\<[E]%
\ColumnHook
\end{hscode}\resethooks
}
%
An \ensuremath{\mathsf{Edge}\;\VarSym{\alpha}} is given by two vertices of type \ensuremath{\VarSym{\alpha}}; the \ensuremath{\mathkw{deriving}\;\mathsf{Eq}} statement
generates the code that is necessary to determine whether or not two \ensuremath{\mathsf{Edge}}s are
equal using the \ensuremath{(\equiv )} operator. In order to perform our analysis, we should wish
to transform the raw list of edges into a tree structure. The basic procedure is
as follows:

First, we try to find the root vertex of the tree. This will be a vertex that is
given as the head of one of the words, but does not itself appear in the
sentence:

\begin{hscode}\SaveRestoreHook
\column{B}{@{}>{\hspre}l<{\hspost}@{}}%
\column{3}{@{}>{\hspre}l<{\hspost}@{}}%
\column{5}{@{}>{\hspre}l<{\hspost}@{}}%
\column{12}{@{}>{\hspre}l<{\hspost}@{}}%
\column{E}{@{}>{\hspre}l<{\hspost}@{}}%
\>[3]{}\FN{rootVertex}\ConSym{::}\mathsf{Eq}\;\VarSym{\alpha}\Rightarrow [\mskip1.5mu \mathsf{Edge}\;\VarSym{\alpha}\mskip1.5mu]\to \mathsf{Maybe}\;\VarSym{\alpha}{}\<[E]%
\\
\>[3]{}\FN{rootVertex}\;\Varid{es}\mathrel{=}\FN{find}\;(\notin \FN{deps})\;\FN{heads}\;\mathkw{where}{}\<[E]%
\\
\>[3]{}\hsindent{2}{}\<[5]%
\>[5]{}\FN{heads}{}\<[12]%
\>[12]{}\mathrel{=}\llbracket\;(\lambda (\Varid{x}\leftrightarrow\Varid{y})\to \Varid{x})\;\Varid{es}\;\rrbracket{}\<[E]%
\\
\>[3]{}\hsindent{2}{}\<[5]%
\>[5]{}\FN{deps}{}\<[12]%
\>[12]{}\mathrel{=}\llbracket\;(\lambda (\Varid{x}\leftrightarrow\Varid{y})\to \Varid{y})\;\Varid{es}\;\rrbracket{}\<[E]%
\ColumnHook
\end{hscode}\resethooks
If the data that we are working with are not well-formed, there is a chance that
we will not find a root vertex; that is why the type is given as \ensuremath{\mathsf{Maybe}}.

Then, given a root vertex, we look to find all the edges that
it touches, and try to build the subtrees that are connected with those edges.
\begin{hscode}\SaveRestoreHook
\column{B}{@{}>{\hspre}l<{\hspost}@{}}%
\column{3}{@{}>{\hspre}l<{\hspost}@{}}%
\column{5}{@{}>{\hspre}l<{\hspost}@{}}%
\column{18}{@{}>{\hspre}l<{\hspost}@{}}%
\column{E}{@{}>{\hspre}l<{\hspost}@{}}%
\>[3]{}\FN{onEdge}\ConSym{::}\mathsf{Eq}\;\VarSym{\alpha}\Rightarrow \VarSym{\alpha}\to \mathsf{Edge}\;\VarSym{\alpha}\to \mathbb{B}{}\<[E]%
\\
\>[3]{}\FN{onEdge}\;\Varid{i}\;(\Varid{x}\leftrightarrow\Varid{y})\mathrel{=}\Varid{x}\equiv \Varid{i}\mathrel{\vee}\Varid{y}\equiv \Varid{i}{}\<[E]%
\\[\blanklineskip]%
\>[3]{}\FN{oppositeVertex}\ConSym{::}\mathsf{Eq}\;\VarSym{\alpha}\Rightarrow \VarSym{\alpha}\to \mathsf{Edge}\;\VarSym{\alpha}\to \VarSym{\alpha}{}\<[E]%
\\
\>[3]{}\FN{oppositeVertex}\;\Varid{i}\;(\Varid{x}\leftrightarrow\Varid{y}){}\<[E]%
\\
\>[3]{}\hsindent{2}{}\<[5]%
\>[5]{}\mid \Varid{x}\equiv \Varid{i}{}\<[18]%
\>[18]{}\mathrel{=}\Varid{y}{}\<[E]%
\\
\>[3]{}\hsindent{2}{}\<[5]%
\>[5]{}\mid \Varid{otherwise}{}\<[18]%
\>[18]{}\mathrel{=}\Varid{x}{}\<[E]%
\ColumnHook
\end{hscode}\resethooks
This is done recursively until the list of edges is exhausted and we have a
complete tree structure:

\begin{hscode}\SaveRestoreHook
\column{B}{@{}>{\hspre}l<{\hspost}@{}}%
\column{3}{@{}>{\hspre}l<{\hspost}@{}}%
\column{5}{@{}>{\hspre}l<{\hspost}@{}}%
\column{7}{@{}>{\hspre}l<{\hspost}@{}}%
\column{24}{@{}>{\hspre}l<{\hspost}@{}}%
\column{E}{@{}>{\hspre}l<{\hspost}@{}}%
\>[3]{}\FN{treeFromEdges}\ConSym{::}\mathsf{Ord}\;\VarSym{\alpha}\Rightarrow [\mskip1.5mu \mathsf{Edge}\;\VarSym{\alpha}\mskip1.5mu]\to \mathsf{Maybe}\;(\mathsf{Tree}\;\VarSym{\alpha}){}\<[E]%
\\
\>[3]{}\FN{treeFromEdges}\;\Varid{es}\mathrel{=}\llbracket\;(\FN{buildWithRoot}\;\Varid{es})\;(\FN{rootVertex}\;\Varid{es})\;\rrbracket\;\mathkw{where}{}\<[E]%
\\
\>[3]{}\hsindent{2}{}\<[5]%
\>[5]{}\FN{buildWithRoot}\;\Varid{es}\;\Varid{root}\mathrel{=}\Varid{root}\curvearrowright\FN{sortedChildren}\;\mathkw{where}{}\<[E]%
\\
\>[5]{}\hsindent{2}{}\<[7]%
\>[7]{}\FN{roots}{}\<[24]%
\>[24]{}\mathrel{=}\llbracket\;(\FN{oppositeVertex}\;\Varid{root})\;\FN{localVertices}\;\rrbracket{}\<[E]%
\\
\>[5]{}\hsindent{2}{}\<[7]%
\>[7]{}\FN{children}{}\<[24]%
\>[24]{}\mathrel{=}\llbracket\;(\FN{buildWithRoot}\;\FN{foreignVertices})\;\FN{roots}\;\rrbracket{}\<[E]%
\\
\>[5]{}\hsindent{2}{}\<[7]%
\>[7]{}\FN{localVertices}{}\<[24]%
\>[24]{}\mathrel{=}\FN{filter}\;(\FN{onEdge}\;\Varid{root})\;\Varid{es}{}\<[E]%
\\
\>[5]{}\hsindent{2}{}\<[7]%
\>[7]{}\FN{foreignVertices}{}\<[24]%
\>[24]{}\mathrel{=}\FN{filter}\;(\neg \circ\FN{onEdge}\;\Varid{root})\;\Varid{es}{}\<[E]%
\\
\>[5]{}\hsindent{2}{}\<[7]%
\>[7]{}\FN{sortedChildren}{}\<[24]%
\>[24]{}\mathrel{=}\FN{sortBy}\;(\FN{compare}\mathbin{`\FN{on}`}\FN{getLabel})\;\FN{children}{}\<[E]%
\ColumnHook
\end{hscode}\resethooks

\subsection{Counting Violations: Computing \ensuremath{\FN{\omega}}}
\label{sec:counting}

Violations are given as an integer tally:
\begin{hscode}\SaveRestoreHook
\column{B}{@{}>{\hspre}l<{\hspost}@{}}%
\column{3}{@{}>{\hspre}l<{\hspost}@{}}%
\column{E}{@{}>{\hspre}l<{\hspost}@{}}%
\>[3]{}\mathkw{type}\;\mathsf{Violations}\mathrel{=}\mathbb{Z}{}\<[E]%
\ColumnHook
\end{hscode}\resethooks
The basic procedure for counting projectivity violations is as follows: flatten
down the tree into a list of edges cross-referenced by their vertical position
in the tree; then traverse the list and see how many times these edges intersect
each other.
\begin{hscode}\SaveRestoreHook
\column{B}{@{}>{\hspre}l<{\hspost}@{}}%
\column{3}{@{}>{\hspre}l<{\hspost}@{}}%
\column{E}{@{}>{\hspre}l<{\hspost}@{}}%
\>[3]{}\mathkw{type}\;\mathsf{Level}\mathrel{=}\mathbb{Z}{}\<[E]%
\ColumnHook
\end{hscode}\resethooks
The vertical position of a node in a tree is represented as its \ensuremath{\mathsf{Level}},
counting backwards from the total depth of the tree. That is, the deepest node
in the tree is at level \ensuremath{\Numeral{0}}, and the highest node in the tree is at level \ensuremath{\Varid{n}},
where \ensuremath{\Varid{n}} is the tree's depth.
\begin{hscode}\SaveRestoreHook
\column{B}{@{}>{\hspre}l<{\hspost}@{}}%
\column{3}{@{}>{\hspre}l<{\hspost}@{}}%
\column{15}{@{}>{\hspre}l<{\hspost}@{}}%
\column{E}{@{}>{\hspre}l<{\hspost}@{}}%
\>[3]{}\FN{levels}\ConSym{::}\mathsf{Tree}\;\VarSym{\alpha}\to [\mskip1.5mu [\mskip1.5mu \VarSym{\alpha}\mskip1.5mu]\mskip1.5mu]{}\<[E]%
\\
\>[3]{}\FN{levels}\;\Varid{t}\mathrel{=}\FN{fmap}\;(\FN{fmap}\;\FN{getLabel})\mathbin{\$}{}\<[E]%
\\
\>[3]{}\hsindent{12}{}\<[15]%
\>[15]{}\FN{takeWhile}\;(\neg \circ\FN{null})\mathbin{\$}{}\<[E]%
\\
\>[3]{}\hsindent{12}{}\<[15]%
\>[15]{}\FN{iterate}\;(\bind \FN{getForest})\;[\mskip1.5mu \Varid{t}\mskip1.5mu]{}\<[E]%
\ColumnHook
\end{hscode}\resethooks
\begin{hscode}\SaveRestoreHook
\column{B}{@{}>{\hspre}l<{\hspost}@{}}%
\column{3}{@{}>{\hspre}l<{\hspost}@{}}%
\column{E}{@{}>{\hspre}l<{\hspost}@{}}%
\>[3]{}\FN{depth}\ConSym{::}\mathsf{Tree}\;\VarSym{\alpha}\to \mathbb{Z}{}\<[E]%
\\
\>[3]{}\FN{depth}\mathrel{=}\FN{length}\circ\FN{levels}{}\<[E]%
\ColumnHook
\end{hscode}\resethooks
We can now annotate each node in a tree with what level it is at:
\begin{hscode}\SaveRestoreHook
\column{B}{@{}>{\hspre}l<{\hspost}@{}}%
\column{3}{@{}>{\hspre}l<{\hspost}@{}}%
\column{5}{@{}>{\hspre}l<{\hspost}@{}}%
\column{E}{@{}>{\hspre}l<{\hspost}@{}}%
\>[3]{}\FN{annotateLevels}\ConSym{::}\mathsf{Tree}\;\VarSym{\alpha}\to \mathsf{Tree}\;(\mathsf{Level},\VarSym{\alpha}){}\<[E]%
\\
\>[3]{}\FN{annotateLevels}\;\Varid{tree}\mathrel{=}\FN{aux}\;(\FN{depth}\;\Varid{tree})\;\Varid{tree}\;\mathkw{where}{}\<[E]%
\\
\>[3]{}\hsindent{2}{}\<[5]%
\>[5]{}\FN{aux}\;\Varid{l}\;(\Varid{x}\curvearrowright\Varid{ts})\mathrel{=}(\Varid{l},\Varid{x})\curvearrowright\llbracket\;(\FN{aux}\;(\Varid{l}\VarSym{-}\Numeral{1}))\;\Varid{ts}\;\rrbracket{}\<[E]%
\ColumnHook
\end{hscode}\resethooks
Then, we fold up the tree into a list of edges and levels:

\begin{hscode}\SaveRestoreHook
\column{B}{@{}>{\hspre}l<{\hspost}@{}}%
\column{3}{@{}>{\hspre}l<{\hspost}@{}}%
\column{5}{@{}>{\hspre}l<{\hspost}@{}}%
\column{7}{@{}>{\hspre}l<{\hspost}@{}}%
\column{E}{@{}>{\hspre}l<{\hspost}@{}}%
\>[3]{}\FN{allEdges}\ConSym{::}\mathsf{Ord}\;\VarSym{\alpha}\Rightarrow \mathsf{Tree}\;\VarSym{\alpha}\to [\mskip1.5mu (\mathsf{Level},\mathsf{Edge}\;\VarSym{\alpha})\mskip1.5mu]{}\<[E]%
\\
\>[3]{}\FN{allEdges}\;\Varid{tree}\mathrel{=}\FN{aux}\;(\FN{annotateLevels}\;\Varid{tree})\;\mathkw{where}{}\<[E]%
\\
\>[3]{}\hsindent{2}{}\<[5]%
\>[5]{}\FN{aux}\;((\anonymous ,\Varid{x})\curvearrowright\Varid{ts})\mathrel{=}\Varid{ts}\bind \FN{go}\;\mathkw{where}{}\<[E]%
\\
\>[5]{}\hsindent{2}{}\<[7]%
\>[7]{}\FN{go}\;\Varid{t}\mathord{@}((\Varid{l},\Varid{y})\curvearrowright\anonymous )\mathrel{=}(\Varid{l},\FN{edgeWithRange}\;[\mskip1.5mu \Varid{x},\Varid{y}\mskip1.5mu])\mathbin{:}\FN{aux}\;\Varid{t}{}\<[E]%
\ColumnHook
\end{hscode}\resethooks
\begin{hscode}\SaveRestoreHook
\column{B}{@{}>{\hspre}l<{\hspost}@{}}%
\column{3}{@{}>{\hspre}l<{\hspost}@{}}%
\column{E}{@{}>{\hspre}l<{\hspost}@{}}%
\>[3]{}\FN{edgeWithRange}\ConSym{::}\mathsf{Ord}\;\VarSym{\alpha}\Rightarrow [\mskip1.5mu \VarSym{\alpha}\mskip1.5mu]\to \mathsf{Edge}\;\VarSym{\alpha}{}\<[E]%
\\
\>[3]{}\FN{edgeWithRange}\;\Varid{xs}\mathrel{=}\FN{minimum}\;\Varid{xs}\leftrightarrow\FN{maximum}\;\Varid{xs}{}\<[E]%
\ColumnHook
\end{hscode}\resethooks
A handy way to think of edges annotated by levels is as a representation of the
arc itself, where the vertices of the edge are the endpoints, and the level is the
height of the arc. Now, we can count the violations that occur between two arcs.
\begin{hscode}\SaveRestoreHook
\column{B}{@{}>{\hspre}l<{\hspost}@{}}%
\column{3}{@{}>{\hspre}l<{\hspost}@{}}%
\column{5}{@{}>{\hspre}l<{\hspost}@{}}%
\column{54}{@{}>{\hspre}l<{\hspost}@{}}%
\column{E}{@{}>{\hspre}l<{\hspost}@{}}%
\>[3]{}\FN{checkEdges}\ConSym{::}\mathsf{Ord}\;\VarSym{\alpha}\Rightarrow (\mathsf{Level},\mathsf{Edge}\;\VarSym{\alpha})\to (\mathsf{Level},\mathsf{Edge}\;\VarSym{\alpha})\to \mathsf{Violations}{}\<[E]%
\\
\>[3]{}\FN{checkEdges}\;(\Varid{l},\Varid{xy}\mathord{@}(\Varid{x}\leftrightarrow\Varid{y}))\;(\Varid{l'},\Varid{uv}\mathord{@}(\Varid{u}\leftrightarrow\Varid{v})){}\<[E]%
\\
\>[3]{}\hsindent{2}{}\<[5]%
\>[5]{}\mid \Varid{x}\in_E\Varid{uv}\mathrel{\wedge}((\Varid{y}\geq \Varid{v}\mathrel{\wedge}\Varid{l}\VarSym{>}\Varid{l'})\mathrel{\vee}\Varid{y}\VarSym{>}\Varid{v}){}\<[54]%
\>[54]{}\mathrel{=}\Numeral{1}{}\<[E]%
\\
\>[3]{}\hsindent{2}{}\<[5]%
\>[5]{}\mid \Varid{y}\in_E\Varid{uv}\mathrel{\wedge}((\Varid{x}\leq \Varid{u}\mathrel{\wedge}\Varid{l}\VarSym{>}\Varid{l'})\mathrel{\vee}\Varid{u}\VarSym{<}\Varid{u}){}\<[54]%
\>[54]{}\mathrel{=}\Numeral{1}{}\<[E]%
\\
\>[3]{}\hsindent{2}{}\<[5]%
\>[5]{}\mid \Varid{u}\in_E\Varid{xy}\mathrel{\wedge}((\Varid{v}\geq \Varid{y}\mathrel{\wedge}\Varid{l}\VarSym{<}\Varid{l'})\mathrel{\vee}\Varid{v}\VarSym{>}\Varid{y}){}\<[54]%
\>[54]{}\mathrel{=}\Numeral{1}{}\<[E]%
\\
\>[3]{}\hsindent{2}{}\<[5]%
\>[5]{}\mid \Varid{v}\in_E\Varid{xy}\mathrel{\wedge}((\Varid{u}\leq \Varid{x}\mathrel{\wedge}\Varid{l}\VarSym{<}\Varid{l'})\mathrel{\vee}\Varid{u}\VarSym{<}\Varid{x}){}\<[54]%
\>[54]{}\mathrel{=}\Numeral{1}{}\<[E]%
\\
\>[3]{}\hsindent{2}{}\<[5]%
\>[5]{}\mid \Varid{x}\in_E\Varid{uv}\mathrel{\wedge}\Varid{y}\in_E\Varid{uv}\mathrel{\wedge}\Varid{l}\geq \Varid{l'}{}\<[54]%
\>[54]{}\mathrel{=}\Numeral{2}{}\<[E]%
\\
\>[3]{}\hsindent{2}{}\<[5]%
\>[5]{}\mid \Varid{u}\in_E\Varid{xy}\mathrel{\wedge}\Varid{v}\in_E\Varid{xy}\mathrel{\wedge}\Varid{l}\leq \Varid{l'}{}\<[54]%
\>[54]{}\mathrel{=}\Numeral{2}{}\<[E]%
\\
\>[3]{}\hsindent{2}{}\<[5]%
\>[5]{}\mid \Varid{otherwise}{}\<[54]%
\>[54]{}\mathrel{=}\Numeral{0}{}\<[E]%
\ColumnHook
\end{hscode}\resethooks
We determine whether a vertex is in the bounds of an edge using \ensuremath{\cdot \in_E\cdot }.
\begin{hscode}\SaveRestoreHook
\column{B}{@{}>{\hspre}l<{\hspost}@{}}%
\column{3}{@{}>{\hspre}l<{\hspost}@{}}%
\column{24}{@{}>{\hspre}c<{\hspost}@{}}%
\column{24E}{@{}l@{}}%
\column{28}{@{}>{\hspre}l<{\hspost}@{}}%
\column{42}{@{}>{\hspre}l<{\hspost}@{}}%
\column{E}{@{}>{\hspre}l<{\hspost}@{}}%
\>[3]{}\cdot \in_E\cdot \ConSym{::}\mathsf{Ord}\;\VarSym{\alpha}\Rightarrow \VarSym{\alpha}\to \mathsf{Edge}\;\VarSym{\alpha}\to \mathbb{B}{}\<[E]%
\\
\>[3]{}\Varid{z}\in_E\Varid{x}\leftrightarrow\Varid{y}{}\<[24]%
\>[24]{}\mathrel{=}{}\<[24E]%
\>[28]{}\Varid{z}\VarSym{>}\FN{minimum}\;{}\<[42]%
\>[42]{}[\mskip1.5mu \Varid{x},\Varid{y}\mskip1.5mu]{}\<[E]%
\\
\>[24]{}\mathrel{\wedge}{}\<[24E]%
\>[28]{}\Varid{z}\VarSym{<}\FN{maximum}\;{}\<[42]%
\>[42]{}[\mskip1.5mu \Varid{x},\Varid{y}\mskip1.5mu]{}\<[E]%
\ColumnHook
\end{hscode}\resethooks
\ignore{
\begin{hscode}\SaveRestoreHook
\column{B}{@{}>{\hspre}l<{\hspost}@{}}%
\column{3}{@{}>{\hspre}l<{\hspost}@{}}%
\column{E}{@{}>{\hspre}l<{\hspost}@{}}%
\>[3]{}\Varid{x}\notin_E\Varid{xy}\mathrel{=}\neg \;(\Varid{x}\in_E\Varid{xy}){}\<[E]%
\ColumnHook
\end{hscode}\resethooks
}
%
We can now use what we've built to count the intersections that occur in a
collection of edges. This is done by adding up the result of \ensuremath{\FN{checkEdges}} of the
combination of each edge with the subset of edges which are at or below its
level:

\begin{hscode}\SaveRestoreHook
\column{B}{@{}>{\hspre}l<{\hspost}@{}}%
\column{3}{@{}>{\hspre}l<{\hspost}@{}}%
\column{5}{@{}>{\hspre}l<{\hspost}@{}}%
\column{25}{@{}>{\hspre}l<{\hspost}@{}}%
\column{E}{@{}>{\hspre}l<{\hspost}@{}}%
\>[3]{}\FN{edgeViolations}\ConSym{::}\mathsf{Ord}\;\VarSym{\alpha}\Rightarrow [\mskip1.5mu (\mathsf{Level},\mathsf{Edge}\;\VarSym{\alpha})\mskip1.5mu]\to \mathsf{Violations}{}\<[E]%
\\
\>[3]{}\FN{edgeViolations}\;\Varid{xs}\mathrel{=}\FN{sum}\;\llbracket\;\FN{violationsWith}\;\Varid{xs}\;\rrbracket\;\mathkw{where}{}\<[E]%
\\
\>[3]{}\hsindent{2}{}\<[5]%
\>[5]{}\FN{rangesBelow}\;(\Varid{l},\anonymous ){}\<[25]%
\>[25]{}\mathrel{=}\FN{filter}\;(\lambda (\Varid{l'},\anonymous )\to \Varid{l'}\leq \Varid{l})\;\Varid{xs}{}\<[E]%
\\
\>[3]{}\hsindent{2}{}\<[5]%
\>[5]{}\FN{violationsWith}\;\Varid{x}{}\<[25]%
\>[25]{}\mathrel{=}\FN{sum}\;\llbracket\;(\FN{checkEdges}\;\Varid{x})\;(\FN{rangesBelow}\;\Varid{x})\;\rrbracket{}\<[E]%
\ColumnHook
\end{hscode}\resethooks
Finally, \ensuremath{\FN{\omega}} is computed for a tree as follows:

\begin{hscode}\SaveRestoreHook
\column{B}{@{}>{\hspre}l<{\hspost}@{}}%
\column{3}{@{}>{\hspre}l<{\hspost}@{}}%
\column{5}{@{}>{\hspre}l<{\hspost}@{}}%
\column{E}{@{}>{\hspre}l<{\hspost}@{}}%
\>[3]{}\FN{\omega}\ConSym{::}\mathsf{Ord}\;\VarSym{\alpha}\Rightarrow \mathsf{Tree}\;\VarSym{\alpha}\to \mathbb{Q}{}\<[E]%
\\
\>[3]{}\FN{\omega}\;\Varid{tree}\mathrel{=}\dfrac{\FN{edgeViolations}\;\FN{edges}}{\FN{length}\;\FN{edges}}\;\mathkw{where}{}\<[E]%
\\
\>[3]{}\hsindent{2}{}\<[5]%
\>[5]{}\FN{edges}\mathrel{=}\FN{allEdges}\;\Varid{tree}{}\<[E]%
\ColumnHook
\end{hscode}\resethooks

\section{Parsing the Perseus Treebank}
\label{sec:parsing}

\noindent
We can express the general shape of a treebank document as follows:

\begin{hscode}\SaveRestoreHook
\column{B}{@{}>{\hspre}l<{\hspost}@{}}%
\column{3}{@{}>{\hspre}l<{\hspost}@{}}%
\column{E}{@{}>{\hspre}l<{\hspost}@{}}%
\>[3]{}\mathkw{type}\;\mathsf{Document}\mathrel{=}[\mskip1.5mu \mathsf{Sentence}\mskip1.5mu]{}\<[E]%
\\
\>[3]{}\mathkw{data}\;\mathsf{Sentence}\mathrel{=}\mathsf{Sentence}\;\{\mskip1.5mu \FN{sentenceId}\ConSym{::}\mathbb{Z},\FN{sentenceEdges}\ConSym{::}[\mskip1.5mu \mathsf{Edge}\;\mathbb{Z}\mskip1.5mu]\mskip1.5mu\}{}\<[E]%
\ColumnHook
\end{hscode}\resethooks
To construct a \ensuremath{\mathsf{Document}} from the contents of an XML file, it suffices to
find all of the sentences.

\begin{hscode}\SaveRestoreHook
\column{B}{@{}>{\hspre}l<{\hspost}@{}}%
\column{3}{@{}>{\hspre}l<{\hspost}@{}}%
\column{5}{@{}>{\hspre}l<{\hspost}@{}}%
\column{E}{@{}>{\hspre}l<{\hspost}@{}}%
\>[3]{}\FN{documentFromXML}\ConSym{::}[\mskip1.5mu \mathsf{Content}\mskip1.5mu]\to \mathsf{Document}{}\<[E]%
\\
\>[3]{}\FN{documentFromXML}\;\Varid{xml}\mathrel{=}\FN{catMaybes}\;\llbracket\;\FN{sentenceFromXML}\;\FN{elems}\;\rrbracket\;\mathkw{where}{}\<[E]%
\\
\>[3]{}\hsindent{2}{}\<[5]%
\>[5]{}\FN{elems}\mathrel{=}\FN{onlyElems}\;\Varid{xml}\bind \FN{findElements}\;(\FN{simpleName}\;\Char{\char34 sentence\char34}){}\<[E]%
\ColumnHook
\end{hscode}\resethooks
\ensuremath{\mathsf{Sentence}}s are got by taking the contents of their \lstinline{id} attribute,
and extracting edges from their children.

\begin{hscode}\SaveRestoreHook
\column{B}{@{}>{\hspre}l<{\hspost}@{}}%
\column{3}{@{}>{\hspre}l<{\hspost}@{}}%
\column{5}{@{}>{\hspre}l<{\hspost}@{}}%
\column{15}{@{}>{\hspre}l<{\hspost}@{}}%
\column{E}{@{}>{\hspre}l<{\hspost}@{}}%
\>[3]{}\FN{sentenceFromXML}\ConSym{::}\mathsf{Element}\to \mathsf{Maybe}\;\mathsf{Sentence}{}\<[E]%
\\
\>[3]{}\FN{sentenceFromXML}\;\Varid{e}\mathrel{=}\llbracket\;\mathsf{Sentence}\;(\FN{readAttr}\;\Char{\char34 id\char34}\;\Varid{e})\;(\FN{pure}\;\FN{edges})\;\rrbracket\;\mathkw{where}{}\<[E]%
\\
\>[3]{}\hsindent{2}{}\<[5]%
\>[5]{}\FN{edges}{}\<[15]%
\>[15]{}\mathrel{=}\FN{catMaybes}\;\llbracket\;\FN{edgeFromXML}\;\FN{children}\;\rrbracket{}\<[E]%
\\
\>[3]{}\hsindent{2}{}\<[5]%
\>[5]{}\FN{children}{}\<[15]%
\>[15]{}\mathrel{=}\FN{findChildren}\;(\FN{simpleName}\;\Char{\char34 word\char34})\;\Varid{e}{}\<[E]%
\ColumnHook
\end{hscode}\resethooks
An edge is got from an element by taking the contents of its \lstinline{id}
attribute with the contents of its \lstinline{head} attribute.
\begin{hscode}\SaveRestoreHook
\column{B}{@{}>{\hspre}l<{\hspost}@{}}%
\column{3}{@{}>{\hspre}l<{\hspost}@{}}%
\column{5}{@{}>{\hspre}l<{\hspost}@{}}%
\column{8}{@{}>{\hspre}l<{\hspost}@{}}%
\column{E}{@{}>{\hspre}l<{\hspost}@{}}%
\>[3]{}\FN{edgeFromXML}\ConSym{::}\mathsf{Element}\to \mathsf{Maybe}\;(\mathsf{Edge}\;\mathbb{Z}){}\<[E]%
\\
\>[3]{}\FN{edgeFromXML}\;\Varid{e}\mathrel{=}{}\<[E]%
\\
\>[3]{}\hsindent{2}{}\<[5]%
\>[5]{}\mathkw{case}\;\FN{findAttr}\;(\FN{simpleName}\;\Char{\char34 form\char34})\;\Varid{e}\;\mathkw{of}{}\<[E]%
\\
\>[5]{}\hsindent{3}{}\<[8]%
\>[8]{}\mathsf{Just}\;\Varid{x}\mid \Varid{x}\in[\mskip1.5mu \Char{\char34 .\char34},\Char{\char34 ,\char34},\Char{\char34 ;\char34},\Char{\char34 :\char34}\mskip1.5mu]\to \mathsf{Nothing}{}\<[E]%
\\
\>[5]{}\hsindent{3}{}\<[8]%
\>[8]{}\Varid{otherwise}\to \llbracket\;(\FN{readAttr}\;\Char{\char34 head\char34}\;\Varid{e})\;\leftrightarrow\;(\FN{readAttr}\;\Char{\char34 id\char34}\;\Varid{e})\;\rrbracket{}\<[E]%
\ColumnHook
\end{hscode}\resethooks
Thence, turn a sentence into a tree by its edges using the machinery from
Section~\ref{sec:edges-to-trees}.
\begin{hscode}\SaveRestoreHook
\column{B}{@{}>{\hspre}l<{\hspost}@{}}%
\column{3}{@{}>{\hspre}l<{\hspost}@{}}%
\column{E}{@{}>{\hspre}l<{\hspost}@{}}%
\>[3]{}\FN{treeFromSentence}\ConSym{::}\mathsf{Sentence}\to \mathsf{Maybe}\;(\mathsf{Tree}\;\mathbb{Z}){}\<[E]%
\\
\>[3]{}\FN{treeFromSentence}\;(\mathsf{Sentence}\;\anonymous \;\Varid{ws})\mathrel{=}\FN{treeFromEdges}\;\Varid{ws}{}\<[E]%
\ColumnHook
\end{hscode}\resethooks
By applying \ensuremath{\FN{treeFromSentence}} to every sentence within a document, we can
generate all the trees in a document.
\begin{hscode}\SaveRestoreHook
\column{B}{@{}>{\hspre}l<{\hspost}@{}}%
\column{3}{@{}>{\hspre}l<{\hspost}@{}}%
\column{E}{@{}>{\hspre}l<{\hspost}@{}}%
\>[3]{}\FN{treesFromDocument}\ConSym{::}\mathsf{Document}\to [\mskip1.5mu \mathsf{Tree}\;\mathbb{Z}\mskip1.5mu]{}\<[E]%
\\
\>[3]{}\FN{treesFromDocument}\;\Varid{ss}\mathrel{=}\FN{catMaybes}\;\llbracket\;\FN{treeFromSentence}\;\Varid{ss}\;\rrbracket{}\<[E]%
\ColumnHook
\end{hscode}\resethooks
By combining the above, we also may derive a document structure from a file on
disk.

\begin{hscode}\SaveRestoreHook
\column{B}{@{}>{\hspre}l<{\hspost}@{}}%
\column{3}{@{}>{\hspre}l<{\hspost}@{}}%
\column{E}{@{}>{\hspre}l<{\hspost}@{}}%
\>[3]{}\FN{documentFromFile}\ConSym{::}\mathsf{FilePath}\to \mathsf{IO}\;\mathsf{Document}{}\<[E]%
\\
\>[3]{}\FN{documentFromFile}\;\Varid{path}\mathrel{=}\llbracket\;(\FN{documentFromXML}\circ\FN{parseXML})\;(\FN{readFile}\;\Varid{path})\;\rrbracket{}\<[E]%
\ColumnHook
\end{hscode}\resethooks
\section{Analysis of Data}

We compute the mean \ensuremath{\FN{\omega}} of the trees contained in a document as follows:
\begin{hscode}\SaveRestoreHook
\column{B}{@{}>{\hspre}l<{\hspost}@{}}%
\column{3}{@{}>{\hspre}l<{\hspost}@{}}%
\column{E}{@{}>{\hspre}l<{\hspost}@{}}%
\>[3]{}\FN{analyzeDocument}\ConSym{::}\mathsf{Document}\to \mathbb{Q}{}\<[E]%
\\
\>[3]{}\FN{analyzeDocument}\;\Varid{doc}\mathrel{=}\FN{mean}\;\llbracket\;\FN{\omega}\;(\FN{treesFromDocument}\;\Varid{doc})\;\rrbracket{}\<[E]%
\ColumnHook
\end{hscode}\resethooks
We will wish to compare the \ensuremath{\FN{\omega}} for parts of \emph{Antigone}. A section is
given by a two sentence indices (a beginning and an end):
\begin{hscode}\SaveRestoreHook
\column{B}{@{}>{\hspre}l<{\hspost}@{}}%
\column{3}{@{}>{\hspre}l<{\hspost}@{}}%
\column{E}{@{}>{\hspre}l<{\hspost}@{}}%
\>[3]{}\mathkw{data}\;\mathsf{Section}\mathrel{=}\mathbb{Z}\cdots\mathbb{Z}{}\<[E]%
\ColumnHook
\end{hscode}\resethooks
Then, the entire document can be cut down into smaller documents by section:

\begin{hscode}\SaveRestoreHook
\column{B}{@{}>{\hspre}l<{\hspost}@{}}%
\column{3}{@{}>{\hspre}l<{\hspost}@{}}%
\column{5}{@{}>{\hspre}l<{\hspost}@{}}%
\column{E}{@{}>{\hspre}l<{\hspost}@{}}%
\>[3]{}\FN{restrictDocument}\ConSym{::}\mathsf{Section}\to \mathsf{Document}\to \mathsf{Document}{}\<[E]%
\\
\>[3]{}\FN{restrictDocument}\;(\Varid{start}\cdots\Varid{finish})\mathrel{=}\FN{filter}\;\FN{withinSection}\;\mathkw{where}{}\<[E]%
\\
\>[3]{}\hsindent{2}{}\<[5]%
\>[5]{}\FN{withinSection}\;(\mathsf{Sentence}\;\Varid{i}\;\anonymous )\mathrel{=}\Varid{i}\geq \Varid{start}\mathrel{\wedge}\Varid{i}\leq \Varid{finish}{}\<[E]%
\ColumnHook
\end{hscode}\resethooks
\ignore{
\begin{hscode}\SaveRestoreHook
\column{B}{@{}>{\hspre}l<{\hspost}@{}}%
\column{3}{@{}>{\hspre}l<{\hspost}@{}}%
\column{5}{@{}>{\hspre}l<{\hspost}@{}}%
\column{17}{@{}>{\hspre}l<{\hspost}@{}}%
\column{E}{@{}>{\hspre}l<{\hspost}@{}}%
\>[3]{}\Varid{makeTable}\ConSym{::}[\mskip1.5mu (\mathsf{Section},\mathsf{Section},\mathsf{String})\mskip1.5mu]\to \mathsf{IO}\;\mathsf{UnquotedString}{}\<[E]%
\\
\>[3]{}\Varid{makeTable}\;\Varid{sections}\mathrel{=}\mathkw{do}{}\<[E]%
\\
\>[3]{}\hsindent{2}{}\<[5]%
\>[5]{}\Varid{antigone}\leftarrow \FN{documentFromFile}\;\Char{\char34 antigone.xml\char34}{}\<[E]%
\\
\>[3]{}\hsindent{2}{}\<[5]%
\>[5]{}\mathkw{let}\;\Varid{pre}\mathrel{=}\Char{\char34 \char92 \char92 begin\char123 tabular\char125 \char123 clc\char125 \char92 \char92 toprule\char92 \char92 textbf\char123 Lines\char125 \&\char92 \char92 textbf\char123 \char125 \&|omega|\char92 \char92 \char92 \char92 ~\char92 \char92 midrule\char34}{}\<[E]%
\\
\>[3]{}\hsindent{2}{}\<[5]%
\>[5]{}\mathkw{let}\;\Varid{post}\mathrel{=}\Char{\char34 \char92 \char92 bottomrule\char92 \char92 end\char123 tabular\char125 \char34}{}\<[E]%
\\
\>[3]{}\hsindent{2}{}\<[5]%
\>[5]{}\mathkw{let}\;\Varid{omegas}\mathrel{=}(\lambda (\anonymous ,\Varid{r},\anonymous )\to \FN{analyzeDocument}\mathbin{\$}\FN{restrictDocument}\;\Varid{r}\;\Varid{antigone})\;\mathop{\langle\$\rangle}\;\Varid{sections}{}\<[E]%
\\
\>[3]{}\hsindent{2}{}\<[5]%
\>[5]{}\mathkw{let}\;\Varid{body}\mathrel{=}\FN{fold}\mathbin{\$}\FN{uncurry}\;\Varid{makeTableRow}\;\mathop{\langle\$\rangle}\;\Varid{zip}\;\Varid{sections}\;\Varid{omegas}{}\<[E]%
\\
\>[3]{}\hsindent{2}{}\<[5]%
\>[5]{}\mathkw{let}\;\Varid{avg}\mathrel{=}\Char{\char34 \char92 \char92 midrule\char92 \char92 multicolumn\char123 3\char125 \char123 r\char125 \char123 |mean~omega|~=~|\char34}\plus \Varid{showRational}\;(\FN{mean}\;\Varid{omegas}){}\<[E]%
\\
\>[5]{}\hsindent{12}{}\<[17]%
\>[17]{}\plus \Char{\char34 ~|,~|sdev|~=~|\char34}\plus \Varid{showRational}\;(\FN{sdev}\;(\Varid{fromRational}\;\mathop{\langle\$\rangle}\;\Varid{omegas}))\plus \Char{\char34 |\char125 \char92 \char92 \char92 \char92 \char34}{}\<[E]%
\\
\>[3]{}\hsindent{2}{}\<[5]%
\>[5]{}\FN{return}\circ\mathsf{Unquote}\mathbin{\$}\Varid{pre}\plus \Varid{body}\plus \Varid{avg}\plus \Varid{post}{}\<[E]%
\ColumnHook
\end{hscode}\resethooks
\begin{hscode}\SaveRestoreHook
\column{B}{@{}>{\hspre}l<{\hspost}@{}}%
\column{3}{@{}>{\hspre}l<{\hspost}@{}}%
\column{E}{@{}>{\hspre}l<{\hspost}@{}}%
\>[3]{}\Varid{showRational}\;\Varid{x}\mathrel{=}\Varid{printf}\;\Char{\char34 \%.2f\char34}\;((\Varid{fromInteger}\mathbin{\$}\Varid{round}\mathbin{\$}\Varid{x}\VarSym{*}(\Numeral{10} \Numeral{2}))\mathbin{/}(\Numeral{10.0}\;\Numeral{2})\ConSym{::}\mathsf{Float}){}\<[E]%
\ColumnHook
\end{hscode}\resethooks
\begin{hscode}\SaveRestoreHook
\column{B}{@{}>{\hspre}l<{\hspost}@{}}%
\column{3}{@{}>{\hspre}l<{\hspost}@{}}%
\column{5}{@{}>{\hspre}l<{\hspost}@{}}%
\column{E}{@{}>{\hspre}l<{\hspost}@{}}%
\>[3]{}\Varid{makeTableRow}\ConSym{::}(\mathsf{Section},\mathsf{Section},\mathsf{String})\to \mathbb{Q}\to \mathsf{String}{}\<[E]%
\\
\>[3]{}\Varid{makeTableRow}\;(\Varid{ls},\Varid{r},\Varid{d})\;\Varid{om}\mathrel{=}\Varid{lines}\plus \Char{\char34 \&\char34}\plus \Varid{desc}\plus \Char{\char34 \&\char34}\plus \FN{\omega}\plus \Char{\char34 \char92 \char92 \char92 \char92 \char34}\;\mathkw{where}{}\<[E]%
\\
\>[3]{}\hsindent{2}{}\<[5]%
\>[5]{}\Varid{desc}\mathrel{=}\Char{\char34 \char92 \char92 emph\char123 \char34}\plus \Varid{d}\plus \Char{\char34 \char125 \char34}{}\<[E]%
\\
\>[3]{}\hsindent{2}{}\<[5]%
\>[5]{}\Varid{lines}\mathrel{=}\Char{\char34 |\char34}\plus \Varid{show}\;\Varid{ls}\plus \Char{\char34 |\char34}{}\<[E]%
\\
\>[3]{}\hsindent{2}{}\<[5]%
\>[5]{}\Varid{range}\mathrel{=}\Char{\char34 |\char34}\plus \Varid{show}\;\Varid{r}\plus \Char{\char34 |\char34}{}\<[E]%
\\
\>[3]{}\hsindent{2}{}\<[5]%
\>[5]{}\FN{\omega}\mathrel{=}\Char{\char34 |\char34}\plus \Varid{showRational}\;\Varid{om}\plus \Char{\char34 |\char34}{}\<[E]%
\ColumnHook
\end{hscode}\resethooks
\begin{hscode}\SaveRestoreHook
\column{B}{@{}>{\hspre}l<{\hspost}@{}}%
\column{3}{@{}>{\hspre}l<{\hspost}@{}}%
\column{5}{@{}>{\hspre}l<{\hspost}@{}}%
\column{E}{@{}>{\hspre}l<{\hspost}@{}}%
\>[3]{}\mathkw{deriving}\;\mathkw{instance}\;\mathsf{Show}\;\mathsf{Section}{}\<[E]%
\\
\>[3]{}\mathkw{newtype}\;\mathsf{UnquotedString}\mathrel{=}\mathsf{Unquote}\;\mathsf{String}{}\<[E]%
\\
\>[3]{}\mathkw{instance}\;\mathsf{Show}\;\mathsf{UnquotedString}\;\mathkw{where}{}\<[E]%
\\
\>[3]{}\hsindent{2}{}\<[5]%
\>[5]{}\Varid{show}\;(\mathsf{Unquote}\;\Varid{str})\mathrel{=}\Varid{str}{}\<[E]%
\ColumnHook
\end{hscode}\resethooks
\begin{hscode}\SaveRestoreHook
\column{B}{@{}>{\hspre}l<{\hspost}@{}}%
\column{3}{@{}>{\hspre}l<{\hspost}@{}}%
\column{11}{@{}>{\hspre}c<{\hspost}@{}}%
\column{11E}{@{}l@{}}%
\column{14}{@{}>{\hspre}l<{\hspost}@{}}%
\column{35}{@{}>{\hspre}l<{\hspost}@{}}%
\column{E}{@{}>{\hspre}l<{\hspost}@{}}%
\>[3]{}\Varid{odes}\ConSym{::}[\mskip1.5mu (\mathsf{Section},\mathsf{Section},\mathsf{String})\mskip1.5mu]{}\<[E]%
\\
\>[3]{}\Varid{odes}\mathrel{=}{}\<[11]%
\>[11]{}[\mskip1.5mu {}\<[11E]%
\>[14]{}(\Numeral{100}\cdots\Numeral{154},{}\<[35]%
\>[35]{}\Numeral{2900135}\cdots\Numeral{2900144},\Char{\char34 First~choral~ode\char34}),{}\<[E]%
\\
\>[14]{}(\Numeral{332}\cdots\Numeral{375},{}\<[35]%
\>[35]{}\Numeral{2900236}\cdots\Numeral{2900247},\Char{\char34 Second~choral~ode\char34}),{}\<[E]%
\\
\>[14]{}(\Numeral{583}\cdots\Numeral{625},{}\<[35]%
\>[35]{}\Numeral{2900390}\cdots\Numeral{2900402},\Char{\char34 Third~choral~ode\char34}),{}\<[E]%
\\
\>[14]{}(\Numeral{781}\cdots\Numeral{800},{}\<[35]%
\>[35]{}\Numeral{2900496}\cdots\Numeral{2900501},\Char{\char34 Fourth~choral~ode\char34}),{}\<[E]%
\\
\>[14]{}(\Numeral{944}\cdots\Numeral{987},{}\<[35]%
\>[35]{}\Numeral{2900554}\cdots\Numeral{2900566},\Char{\char34 Fifth~choral~ode\char34}),{}\<[E]%
\\
\>[14]{}(\Numeral{1116}\cdots\Numeral{1152},{}\<[35]%
\>[35]{}\Numeral{2900649}\cdots\Numeral{2900654},\Char{\char34 Sixth~choral~ode\char34}){}\<[E]%
\\
\>[11]{}\mskip1.5mu]{}\<[11E]%
\ColumnHook
\end{hscode}\resethooks
\begin{hscode}\SaveRestoreHook
\column{B}{@{}>{\hspre}l<{\hspost}@{}}%
\column{3}{@{}>{\hspre}l<{\hspost}@{}}%
\column{12}{@{}>{\hspre}c<{\hspost}@{}}%
\column{12E}{@{}l@{}}%
\column{14}{@{}>{\hspre}l<{\hspost}@{}}%
\column{35}{@{}>{\hspre}l<{\hspost}@{}}%
\column{E}{@{}>{\hspre}l<{\hspost}@{}}%
\>[3]{}\Varid{kommoi}\mathrel{=}[\mskip1.5mu {}\<[E]%
\\
\>[3]{}\hsindent{11}{}\<[14]%
\>[14]{}(\Numeral{806}\cdots\Numeral{816},{}\<[35]%
\>[35]{}\Numeral{2900503}\cdots\Numeral{2900504},\Char{\char34 Antigone's~Lament\char34}),{}\<[E]%
\\
\>[3]{}\hsindent{11}{}\<[14]%
\>[14]{}(\Numeral{823}\cdots\Numeral{833},{}\<[35]%
\>[35]{}\Numeral{2900506}\cdots\Numeral{2900507},\Char{\char34 Antigone's~Lament~(cntd.)\char34}),{}\<[E]%
\\
\>[3]{}\hsindent{11}{}\<[14]%
\>[14]{}(\Numeral{839}\cdots\Numeral{882},{}\<[35]%
\>[35]{}\Numeral{2900511}\cdots\Numeral{2900526},\Char{\char34 Antigone's~Lament~(cntd.)\char34}),{}\<[E]%
\\
\>[3]{}\hsindent{11}{}\<[14]%
\>[14]{}(\Numeral{1261}\cdots\Numeral{1269},{}\<[35]%
\>[35]{}\Numeral{2900714}\cdots\Numeral{2900716},\Char{\char34 Kreon's~Lament\char34}),{}\<[E]%
\\
\>[3]{}\hsindent{11}{}\<[14]%
\>[14]{}(\Numeral{1283}\cdots\Numeral{1292},{}\<[35]%
\>[35]{}\Numeral{2900725}\cdots\Numeral{2900729},\Char{\char34 Kreon's~Lament~(cntd.)\char34}),{}\<[E]%
\\
\>[3]{}\hsindent{11}{}\<[14]%
\>[14]{}(\Numeral{1306}\cdots\Numeral{1311},{}\<[35]%
\>[35]{}\Numeral{2900737}\cdots\Numeral{2900739},\Char{\char34 Kreon's~Lament~(cntd.)\char34}),{}\<[E]%
\\
\>[3]{}\hsindent{11}{}\<[14]%
\>[14]{}(\Numeral{1317}\cdots\Numeral{1325},{}\<[35]%
\>[35]{}\Numeral{2900743}\cdots\Numeral{2900745},\Char{\char34 Kreon's~Lament~(cntd.)\char34}),{}\<[E]%
\\
\>[3]{}\hsindent{11}{}\<[14]%
\>[14]{}(\Numeral{1239}\cdots\Numeral{1246},{}\<[35]%
\>[35]{}\Numeral{2900756}\cdots\Numeral{2900767},\Char{\char34 Kreon's~Lament~(cntd.)\char34}){}\<[E]%
\\
\>[3]{}\hsindent{9}{}\<[12]%
\>[12]{}\mskip1.5mu]{}\<[12E]%
\ColumnHook
\end{hscode}\resethooks
\begin{hscode}\SaveRestoreHook
\column{B}{@{}>{\hspre}l<{\hspost}@{}}%
\column{3}{@{}>{\hspre}l<{\hspost}@{}}%
\column{16}{@{}>{\hspre}c<{\hspost}@{}}%
\column{16E}{@{}l@{}}%
\column{19}{@{}>{\hspre}l<{\hspost}@{}}%
\column{40}{@{}>{\hspre}l<{\hspost}@{}}%
\column{E}{@{}>{\hspre}l<{\hspost}@{}}%
\>[3]{}\Varid{anapaests}\ConSym{::}[\mskip1.5mu (\mathsf{Section},\mathsf{Section},\mathsf{String})\mskip1.5mu]{}\<[E]%
\\
\>[3]{}\Varid{anapaests}\mathrel{=}{}\<[16]%
\>[16]{}[\mskip1.5mu {}\<[16E]%
\>[19]{}(\Numeral{155}\cdots\Numeral{161},{}\<[40]%
\>[40]{}\Numeral{2900145}\cdots\Numeral{2900145},\Char{\char34 Kreon's~Entrance\char34}),{}\<[E]%
\\
\>[19]{}(\Numeral{376}\cdots\Numeral{383},{}\<[40]%
\>[40]{}\Numeral{2900248}\cdots\Numeral{2900251},\Char{\char34 Antigone's~Entrance\char34}),\Comment{-{}-\enskip sung}{}\<[E]%
\\
\>[19]{}(\Numeral{526}\cdots\Numeral{530},{}\<[40]%
\>[40]{}\Numeral{2900345}\cdots\Numeral{2900346},\Char{\char34 Ismene's~Entrance\char34}),{}\<[E]%
\\
\>[19]{}(\Numeral{626}\cdots\Numeral{630},{}\<[40]%
\>[40]{}\Numeral{2900403}\cdots\Numeral{2900404},\Char{\char34 Haimon's~Entrance\char34}),{}\<[E]%
\\
\>[19]{}(\Numeral{801}\cdots\Numeral{805},{}\<[40]%
\>[40]{}\Numeral{2900502}\cdots\Numeral{2900502},\Char{\char34 Antigone's~Entrance\char34}),{}\<[E]%
\\
\>[19]{}(\Numeral{817}\cdots\Numeral{822},{}\<[40]%
\>[40]{}\Numeral{2900505}\cdots\Numeral{2900505},\Char{\char34 Chorus~to~Antigone\char34}),{}\<[E]%
\\
\>[19]{}(\Numeral{834}\cdots\Numeral{838},{}\<[40]%
\>[40]{}\Numeral{2900508}\cdots\Numeral{2900510},\Char{\char34 Chorus~to~Antigone\char34}),{}\<[E]%
\\
\>[19]{}(\Numeral{929}\cdots\Numeral{943},{}\<[40]%
\>[40]{}\Numeral{2900548}\cdots\Numeral{2900553},\Char{\char34 Chorus,~Kreon~and~Antigone\char34}),{}\<[E]%
\\
\>[19]{}(\Numeral{1257}\cdots\Numeral{1260},{}\<[40]%
\>[40]{}\Numeral{2900713}\cdots\Numeral{2900713},\Char{\char34 Chorus~before~Kreon's~Kommos\char34}),{}\<[E]%
\\
\>[19]{}(\Numeral{1347}\cdots\Numeral{1353},{}\<[40]%
\>[40]{}\Numeral{2900758}\cdots\Numeral{2900760},\Char{\char34 Final~anapaests~of~the~Chorus\char34}){}\<[E]%
\\
\>[16]{}\mskip1.5mu]{}\<[16E]%
\ColumnHook
\end{hscode}\resethooks
\begin{hscode}\SaveRestoreHook
\column{B}{@{}>{\hspre}l<{\hspost}@{}}%
\column{3}{@{}>{\hspre}l<{\hspost}@{}}%
\column{15}{@{}>{\hspre}c<{\hspost}@{}}%
\column{15E}{@{}l@{}}%
\column{18}{@{}>{\hspre}l<{\hspost}@{}}%
\column{39}{@{}>{\hspre}l<{\hspost}@{}}%
\column{E}{@{}>{\hspre}l<{\hspost}@{}}%
\>[3]{}\Varid{speeches}\ConSym{::}[\mskip1.5mu (\mathsf{Section},\mathsf{Section},\mathsf{String})\mskip1.5mu]{}\<[E]%
\\
\>[3]{}\Varid{speeches}\mathrel{=}{}\<[15]%
\>[15]{}[\mskip1.5mu {}\<[15E]%
\>[18]{}(\Numeral{162}\cdots\Numeral{210},{}\<[39]%
\>[39]{}\Numeral{2900146}\cdots\Numeral{2900157},\Char{\char34 \char92 \char92 emph\char123 Kreon:\char125 ~ἄνδρες,~τὰ~μὲν~δὴ...\char34}),{}\<[E]%
\\
\>[18]{}(\Numeral{249}\cdots\Numeral{277},{}\<[39]%
\>[39]{}\Numeral{2900191}\cdots\Numeral{2900204},\Char{\char34 \char92 \char92 emph\char123 Guard:\char125 ~οὐκ~οἶδ'·~ἐκεῖ~γὰρ~οὔτε...\char34}),{}\<[E]%
\\
\>[18]{}(\Numeral{280}\cdots\Numeral{314},{}\<[39]%
\>[39]{}\Numeral{2900206}\cdots\Numeral{2900220},\Char{\char34 \char92 \char92 emph\char123 Kreon:\char125 ~παῦσαι,~πρὶν~ὀργῆς...\char34}),{}\<[E]%
\\
\>[18]{}(\Numeral{407}\cdots\Numeral{440},{}\<[39]%
\>[39]{}\Numeral{2900271}\cdots\Numeral{2900282},\Char{\char34 \char92 \char92 emph\char123 Guard:\char125 ~τοιοῦτον~ἦν~τὸ~πρᾶγμ'...\char34}),{}\<[E]%
\\
\>[18]{}(\Numeral{450}\cdots\Numeral{470},{}\<[39]%
\>[39]{}\Numeral{2900291}\cdots\Numeral{2900302},\Char{\char34 \char92 \char92 emph\char123 Antigone:\char125 ~οὐ~γάρ~τί~μοι~Ζεὺς...\char34}),{}\<[E]%
\\
\>[18]{}(\Numeral{473}\cdots\Numeral{495},{}\<[39]%
\>[39]{}\Numeral{2900305}\cdots\Numeral{2900316},\Char{\char34 \char92 \char92 emph\char123 Kreon:\char125 ~ἀλλ'~ἴσθι~τοι...\char34}),{}\<[E]%
\\
\>[18]{}(\Numeral{639}\cdots\Numeral{680},{}\<[39]%
\>[39]{}\Numeral{2900410}\cdots\Numeral{2900427},\Char{\char34 \char92 \char92 emph\char123 Kreon:\char125 ~οὕτω~γὰρ,~ὦ~παῖ...\char34}),{}\<[E]%
\\
\>[18]{}(\Numeral{683}\cdots\Numeral{723},{}\<[39]%
\>[39]{}\Numeral{2900429}\cdots\Numeral{2900446},\Char{\char34 \char92 \char92 emph\char123 Haimon:\char125 ~πἀτερ,~θεοὶ~φύουσιν...\char34}),{}\<[E]%
\\
\>[18]{}(\Numeral{891}\cdots\Numeral{928},{}\<[39]%
\>[39]{}\Numeral{2900531}\cdots\Numeral{2900547},\Char{\char34 \char92 \char92 emph\char123 Antigone:\char125 ~ὦ~τύμβος,~ὦ~νυμφεῖον...\char34}),{}\<[E]%
\\
\>[18]{}(\Numeral{998}\cdots\Numeral{1032},{}\<[39]%
\>[39]{}\Numeral{2900577}\cdots\Numeral{2900595},\Char{\char34 \char92 \char92 emph\char123 Teiresias:\char125 ~γνώσῃ,~τέχνης~σημεῖα...\char34}),{}\<[E]%
\\
\>[18]{}(\Numeral{1033}\cdots\Numeral{1047},{}\<[39]%
\>[39]{}\Numeral{2900596}\cdots\Numeral{2900601},\Char{\char34 \char92 \char92 emph\char123 Kreon:\char125 ~ὦ~πρέσβυ,~πάντες...\char34}),{}\<[E]%
\\
\>[18]{}(\Numeral{1064}\cdots\Numeral{1090},{}\<[39]%
\>[39]{}\Numeral{2900621}\cdots\Numeral{2900628},\Char{\char34 \char92 \char92 emph\char123 Teiresias:\char125 ~ἀλλ'~εὖ~γέ~τοι...\char34}),{}\<[E]%
\\
\>[18]{}(\Numeral{1155}\cdots\Numeral{1172},{}\<[39]%
\>[39]{}\Numeral{2900655}\cdots\Numeral{2900662},\Char{\char34 \char92 \char92 emph\char123 Messenger:\char125 ~Κάδμου~πάροικοι~καὶ...\char34}),{}\<[E]%
\\
\>[18]{}(\Numeral{1192}\cdots\Numeral{1243},{}\<[39]%
\>[39]{}\Numeral{2900681}\cdots\Numeral{2900703},\Char{\char34 \char92 \char92 emph\char123 Messenger:\char125 ~ἐγώ,~φίλη~δέσποινα...\char34})\mskip1.5mu]{}\<[E]%
\ColumnHook
\end{hscode}\resethooks
\begin{hscode}\SaveRestoreHook
\column{B}{@{}>{\hspre}l<{\hspost}@{}}%
\column{3}{@{}>{\hspre}l<{\hspost}@{}}%
\column{14}{@{}>{\hspre}c<{\hspost}@{}}%
\column{14E}{@{}l@{}}%
\column{17}{@{}>{\hspre}l<{\hspost}@{}}%
\column{38}{@{}>{\hspre}l<{\hspost}@{}}%
\column{E}{@{}>{\hspre}l<{\hspost}@{}}%
\>[3]{}\Varid{stichos}\ConSym{::}[\mskip1.5mu (\mathsf{Section},\mathsf{Section},\mathsf{String})\mskip1.5mu]{}\<[E]%
\\
\>[3]{}\Varid{stichos}\mathrel{=}{}\<[14]%
\>[14]{}[\mskip1.5mu {}\<[14E]%
\>[17]{}(\Numeral{536}\cdots\Numeral{576},{}\<[38]%
\>[38]{}\Numeral{2900348}\cdots\Numeral{2900385},\Char{\char34 Ismene,~Antigone~and~Kreon\char34}),{}\<[E]%
\\
\>[17]{}(\Numeral{728}\cdots\Numeral{757},{}\<[38]%
\>[38]{}\Numeral{2900450}\cdots\Numeral{2900480},\Char{\char34 Haimon~and~Kreon\char34}),{}\<[E]%
\\
\>[17]{}(\Numeral{991}\cdots\Numeral{997},{}\<[38]%
\>[38]{}\Numeral{2900569}\cdots\Numeral{2900576},\Char{\char34 Kreon~and~Teiresias\char34}),{}\<[E]%
\\
\>[17]{}(\Numeral{1047}\cdots\Numeral{1063},{}\<[38]%
\>[38]{}\Numeral{2900602}\cdots\Numeral{2900620},\Char{\char34 Kreon~and~Teiresias\char34}),{}\<[E]%
\\
\>[17]{}(\Numeral{1172}\cdots\Numeral{1179},{}\<[38]%
\>[38]{}\Numeral{2900663}\cdots\Numeral{2900674},\Char{\char34 Chorus~and~Messenger\char34}){}\<[E]%
\\
\>[14]{}\mskip1.5mu]{}\<[14E]%
\ColumnHook
\end{hscode}\resethooks
}


\section*{Auxiliary Functions}
\begin{hscode}\SaveRestoreHook
\column{B}{@{}>{\hspre}l<{\hspost}@{}}%
\column{3}{@{}>{\hspre}l<{\hspost}@{}}%
\column{E}{@{}>{\hspre}l<{\hspost}@{}}%
\>[3]{}\FN{simpleName}\ConSym{::}\mathsf{String}\to \mathsf{QName}{}\<[E]%
\\
\>[3]{}\FN{simpleName}\;\Varid{s}\mathrel{=}\mathsf{QName}\;\Varid{s}\;\mathsf{Nothing}\;\mathsf{Nothing}{}\<[E]%
\ColumnHook
\end{hscode}\resethooks
\begin{hscode}\SaveRestoreHook
\column{B}{@{}>{\hspre}l<{\hspost}@{}}%
\column{3}{@{}>{\hspre}l<{\hspost}@{}}%
\column{E}{@{}>{\hspre}l<{\hspost}@{}}%
\>[3]{}\FN{readAttr}\ConSym{::}\mathsf{Read}\;\VarSym{\alpha}\Rightarrow \mathsf{String}\to \mathsf{Element}\to \mathsf{Maybe}\;\VarSym{\alpha}{}\<[E]%
\\
\>[3]{}\FN{readAttr}\;\Varid{n}\mathrel{=}\FN{fmap}\;\FN{read}\circ\FN{findAttr}\;(\FN{simpleName}\;\Varid{n}){}\<[E]%
\ColumnHook
\end{hscode}\resethooks
\begin{hscode}\SaveRestoreHook
\column{B}{@{}>{\hspre}l<{\hspost}@{}}%
\column{3}{@{}>{\hspre}l<{\hspost}@{}}%
\column{11}{@{}>{\hspre}l<{\hspost}@{}}%
\column{E}{@{}>{\hspre}l<{\hspost}@{}}%
\>[3]{}\FN{mean}\ConSym{::}\mathsf{Fractional}\;\Varid{n}\Rightarrow [\mskip1.5mu \Varid{n}\mskip1.5mu]\to \Varid{n}{}\<[E]%
\\
\>[3]{}\FN{mean}\mathrel{=}{}\<[11]%
\>[11]{}\llbracket\;\FN{sum}\;\mathbin{/}\;\FN{length}\;\rrbracket{}\<[E]%
\ColumnHook
\end{hscode}\resethooks
\begin{hscode}\SaveRestoreHook
\column{B}{@{}>{\hspre}l<{\hspost}@{}}%
\column{3}{@{}>{\hspre}l<{\hspost}@{}}%
\column{E}{@{}>{\hspre}l<{\hspost}@{}}%
\>[3]{}\FN{sdev}\ConSym{::}\mathsf{Floating}\;\Varid{n}\Rightarrow [\mskip1.5mu \Varid{n}\mskip1.5mu]\to \Varid{n}{}\<[E]%
\\
\>[3]{}\FN{sdev}\;\Varid{xs}\mathrel{=}\sqrt{\dfrac{\FN{sum}\;\llbracket\;(\lambda \Varid{x}\to {\Varid{x}}^{\Numeral{2}})\;\llbracket\;(\VarSym{-}(\FN{mean}\;\Varid{xs})\VarSym{+})\;\Varid{xs}\;\rrbracket\;\rrbracket}{\FN{length}\;\Varid{xs}\VarSym{-}\Numeral{1}}}{}\<[E]%
\ColumnHook
\end{hscode}\resethooks

\ignore{
\begin{hscode}\SaveRestoreHook
\column{B}{@{}>{\hspre}l<{\hspost}@{}}%
\column{3}{@{}>{\hspre}l<{\hspost}@{}}%
\column{E}{@{}>{\hspre}l<{\hspost}@{}}%
\>[3]{}\llbracket\;\cdot \;\cdot \;\cdot \;\rrbracket\mathrel{=}\llbracket\;\cdot \;\cdot \;\cdot \;\rrbracket{}\<[E]%
\\
\>[3]{}\dfrac{\cdot }{\cdot }\mathrel{=}(\mathbin{\%}){}\<[E]%
\\
\>[3]{}\dfrac{\cdot }{\cdot }\mathrel{=}(\mathbin{/}){}\<[E]%
\ColumnHook
\end{hscode}\resethooks
\begin{hscode}\SaveRestoreHook
\column{B}{@{}>{\hspre}l<{\hspost}@{}}%
\column{3}{@{}>{\hspre}l<{\hspost}@{}}%
\column{E}{@{}>{\hspre}l<{\hspost}@{}}%
\>[3]{}{\Varid{x}}^{\Varid{n}}\mathrel{=}\Varid{x} \Varid{n}{}\<[E]%
\ColumnHook
\end{hscode}\resethooks
}

\end{appendices}
\end{document}



%!TEX encoding = UTF-8 Unicode

\documentclass{article}
%% ODER: format ==         = "\mathrel{==}"
%% ODER: format /=         = "\neq "
%
%
\makeatletter
\@ifundefined{lhs2tex.lhs2tex.sty.read}%
  {\@namedef{lhs2tex.lhs2tex.sty.read}{}%
   \newcommand\SkipToFmtEnd{}%
   \newcommand\EndFmtInput{}%
   \long\def\SkipToFmtEnd#1\EndFmtInput{}%
  }\SkipToFmtEnd

\newcommand\ReadOnlyOnce[1]{\@ifundefined{#1}{\@namedef{#1}{}}\SkipToFmtEnd}
\usepackage{amstext}
\usepackage{amssymb}
\usepackage{stmaryrd}
\DeclareFontFamily{OT1}{cmtex}{}
\DeclareFontShape{OT1}{cmtex}{m}{n}
  {<5><6><7><8>cmtex8
   <9>cmtex9
   <10><10.95><12><14.4><17.28><20.74><24.88>cmtex10}{}
\DeclareFontShape{OT1}{cmtex}{m}{it}
  {<-> ssub * cmtt/m/it}{}
\newcommand{\texfamily}{\fontfamily{cmtex}\selectfont}
\DeclareFontShape{OT1}{cmtt}{bx}{n}
  {<5><6><7><8>cmtt8
   <9>cmbtt9
   <10><10.95><12><14.4><17.28><20.74><24.88>cmbtt10}{}
\DeclareFontShape{OT1}{cmtex}{bx}{n}
  {<-> ssub * cmtt/bx/n}{}
\newcommand{\tex}[1]{\text{\texfamily#1}}	% NEU

\newcommand{\Sp}{\hskip.33334em\relax}


\newcommand{\Conid}[1]{\mathit{#1}}
\newcommand{\Varid}[1]{\mathit{#1}}
\newcommand{\anonymous}{\kern0.06em \vbox{\hrule\@width.5em}}
\newcommand{\plus}{\mathbin{+\!\!\!+}}
\newcommand{\bind}{\mathbin{>\!\!\!>\mkern-6.7mu=}}
\newcommand{\rbind}{\mathbin{=\mkern-6.7mu<\!\!\!<}}% suggested by Neil Mitchell
\newcommand{\sequ}{\mathbin{>\!\!\!>}}
\renewcommand{\leq}{\leqslant}
\renewcommand{\geq}{\geqslant}
\usepackage{polytable}

%mathindent has to be defined
\@ifundefined{mathindent}%
  {\newdimen\mathindent\mathindent\leftmargini}%
  {}%

\def\resethooks{%
  \global\let\SaveRestoreHook\empty
  \global\let\ColumnHook\empty}
\newcommand*{\savecolumns}[1][default]%
  {\g@addto@macro\SaveRestoreHook{\savecolumns[#1]}}
\newcommand*{\restorecolumns}[1][default]%
  {\g@addto@macro\SaveRestoreHook{\restorecolumns[#1]}}
\newcommand*{\aligncolumn}[2]%
  {\g@addto@macro\ColumnHook{\column{#1}{#2}}}

\resethooks

\newcommand{\onelinecommentchars}{\quad-{}- }
\newcommand{\commentbeginchars}{\enskip\{-}
\newcommand{\commentendchars}{-\}\enskip}

\newcommand{\visiblecomments}{%
  \let\onelinecomment=\onelinecommentchars
  \let\commentbegin=\commentbeginchars
  \let\commentend=\commentendchars}

\newcommand{\invisiblecomments}{%
  \let\onelinecomment=\empty
  \let\commentbegin=\empty
  \let\commentend=\empty}

\visiblecomments

\newlength{\blanklineskip}
\setlength{\blanklineskip}{0.66084ex}

\newcommand{\hsindent}[1]{\quad}% default is fixed indentation
\let\hspre\empty
\let\hspost\empty
\newcommand{\NB}{\textbf{NB}}
\newcommand{\Todo}[1]{$\langle$\textbf{To do:}~#1$\rangle$}

\EndFmtInput
\makeatother
%
%
%
%
%
%
% This package provides two environments suitable to take the place
% of hscode, called "plainhscode" and "arrayhscode". 
%
% The plain environment surrounds each code block by vertical space,
% and it uses \abovedisplayskip and \belowdisplayskip to get spacing
% similar to formulas. Note that if these dimensions are changed,
% the spacing around displayed math formulas changes as well.
% All code is indented using \leftskip.
%
% Changed 19.08.2004 to reflect changes in colorcode. Should work with
% CodeGroup.sty.
%
\ReadOnlyOnce{polycode.fmt}%
\makeatletter

\newcommand{\hsnewpar}[1]%
  {{\parskip=0pt\parindent=0pt\par\vskip #1\noindent}}

% can be used, for instance, to redefine the code size, by setting the
% command to \small or something alike
\newcommand{\hscodestyle}{}

% The command \sethscode can be used to switch the code formatting
% behaviour by mapping the hscode environment in the subst directive
% to a new LaTeX environment.

\newcommand{\sethscode}[1]%
  {\expandafter\let\expandafter\hscode\csname #1\endcsname
   \expandafter\let\expandafter\endhscode\csname end#1\endcsname}

% "compatibility" mode restores the non-polycode.fmt layout.

\newenvironment{compathscode}%
  {\par\noindent
   \advance\leftskip\mathindent
   \hscodestyle
   \let\\=\@normalcr
   \let\hspre\(\let\hspost\)%
   \pboxed}%
  {\endpboxed\)%
   \par\noindent
   \ignorespacesafterend}

\newcommand{\compaths}{\sethscode{compathscode}}

% "plain" mode is the proposed default.
% It should now work with \centering.
% This required some changes. The old version
% is still available for reference as oldplainhscode.

\newenvironment{plainhscode}%
  {\hsnewpar\abovedisplayskip
   \advance\leftskip\mathindent
   \hscodestyle
   \let\hspre\(\let\hspost\)%
   \pboxed}%
  {\endpboxed%
   \hsnewpar\belowdisplayskip
   \ignorespacesafterend}

\newenvironment{oldplainhscode}%
  {\hsnewpar\abovedisplayskip
   \advance\leftskip\mathindent
   \hscodestyle
   \let\\=\@normalcr
   \(\pboxed}%
  {\endpboxed\)%
   \hsnewpar\belowdisplayskip
   \ignorespacesafterend}

% Here, we make plainhscode the default environment.

\newcommand{\plainhs}{\sethscode{plainhscode}}
\newcommand{\oldplainhs}{\sethscode{oldplainhscode}}
\plainhs

% The arrayhscode is like plain, but makes use of polytable's
% parray environment which disallows page breaks in code blocks.

\newenvironment{arrayhscode}%
  {\hsnewpar\abovedisplayskip
   \advance\leftskip\mathindent
   \hscodestyle
   \let\\=\@normalcr
   \(\parray}%
  {\endparray\)%
   \hsnewpar\belowdisplayskip
   \ignorespacesafterend}

\newcommand{\arrayhs}{\sethscode{arrayhscode}}

% The mathhscode environment also makes use of polytable's parray 
% environment. It is supposed to be used only inside math mode 
% (I used it to typeset the type rules in my thesis).

\newenvironment{mathhscode}%
  {\parray}{\endparray}

\newcommand{\mathhs}{\sethscode{mathhscode}}

% texths is similar to mathhs, but works in text mode.

\newenvironment{texthscode}%
  {\(\parray}{\endparray\)}

\newcommand{\texths}{\sethscode{texthscode}}

% The framed environment places code in a framed box.

\def\codeframewidth{\arrayrulewidth}
\RequirePackage{calc}

\newenvironment{framedhscode}%
  {\parskip=\abovedisplayskip\par\noindent
   \hscodestyle
   \arrayrulewidth=\codeframewidth
   \tabular{@{}|p{\linewidth-2\arraycolsep-2\arrayrulewidth-2pt}|@{}}%
   \hline\framedhslinecorrect\\{-1.5ex}%
   \let\endoflinesave=\\
   \let\\=\@normalcr
   \(\pboxed}%
  {\endpboxed\)%
   \framedhslinecorrect\endoflinesave{.5ex}\hline
   \endtabular
   \parskip=\belowdisplayskip\par\noindent
   \ignorespacesafterend}

\newcommand{\framedhslinecorrect}[2]%
  {#1[#2]}

\newcommand{\framedhs}{\sethscode{framedhscode}}

% The inlinehscode environment is an experimental environment
% that can be used to typeset displayed code inline.

\newenvironment{inlinehscode}%
  {\(\def\column##1##2{}%
   \let\>\undefined\let\<\undefined\let\\\undefined
   \newcommand\>[1][]{}\newcommand\<[1][]{}\newcommand\\[1][]{}%
   \def\fromto##1##2##3{##3}%
   \def\nextline{}}{\) }%

\newcommand{\inlinehs}{\sethscode{inlinehscode}}

% The joincode environment is a separate environment that
% can be used to surround and thereby connect multiple code
% blocks.

\newenvironment{joincode}%
  {\let\orighscode=\hscode
   \let\origendhscode=\endhscode
   \def\endhscode{\def\hscode{\endgroup\def\@currenvir{hscode}\\}\begingroup}
   %\let\SaveRestoreHook=\empty
   %\let\ColumnHook=\empty
   %\let\resethooks=\empty
   \orighscode\def\hscode{\endgroup\def\@currenvir{hscode}}}%
  {\origendhscode
   \global\let\hscode=\orighscode
   \global\let\endhscode=\origendhscode}%

\makeatother
\EndFmtInput
%
%


\usepackage{xcolor}

% Char literal
\colorlet{Char}{black}

% Numeral literal
\colorlet{Numeral}{Char}

% Keyword
\colorlet{Keyword}{red!50!black}

% Module identifier
\colorlet{ModId}{Char}

% Variable identifier and symbol
\colorlet{VarId}{Char}
\colorlet{VarSym}{VarId}

% Data constructor identifier and symbol
\colorlet{ConId}{VarId}
\colorlet{ConSym}{ConId}

% Type variable identifier
\colorlet{TVarId}{green!45!black}

% Type constructor identifier
\colorlet{TConId}{TVarId}
\colorlet{TConSym}{TConId}

% Type class identifier
\colorlet{TClassId}{TVarId}

% Comment
\colorlet{Comment}{blue!45!black}

\newcommand\Char[1]{\textcolor{Char}{\texttt{#1}}}

% Numeral literal
\newcommand\Numeral[1]{\textcolor{Numeral}{\mathsf{#1}}}

% Keyword
\newcommand\Keyword[1]{\textcolor{Keyword}{\textbf{\textsf{#1}}}}

% Module identifier
\newcommand\ModId[1]{\mathord{\textcolor{ModId}{\mathsf{#1}}}}

% Variable identifier and symbol
\newcommand\VarId[1]{\mathord{\textcolor{VarId}{#1}}}
\let\Varid\VarId
\newcommand\VarSym[1]{\mathbin{\textcolor{VarSym}{#1}}}

% Data constructor identifier and symbol
\newcommand\ConId[1]{\mathord{\textcolor{ConId}{\mathsf{#1}}}}
\let\Conid\ConId
\newcommand\ConSym[1]{\mathbin{\textcolor{ConSym}{\mathsf{#1}}}}

% Type variable identifier
\newcommand\TVarId[1]{\mathord{\textcolor{TVarId}{\mathsf{#1}}}}

% Type constructor identifier
\newcommand\TConId[1]{\mathord{\textcolor{TConId}{\mathsf{#1}}}}
% \newcommand\TConSym[1]{\mathbin{#1}}
\newcommand\TConSym[1]{\mathbin{\textcolor{TConSym}{\mathsf{#1}}}}

% Type class identifier
\newcommand\TClassId[1]{\mathord{\textcolor{TClassId}{\textit{\textsf{#1}}}}}

% Comment
\newcommand\Comment[1]{\textcolor{Comment}{\textit{\textsf{#1}}}}

% Package identifier (used in text, not code/math environment)
\newcommand\PkgId[1]{\textcolor{Char}{\texttt{#1}}}




% Comments: one-line, nested, pragmas



\newcommand{\FN}{\mathsf}
\newcommand{\comment}[1]{\marginpar{#1}}
\newcommand{\ignore}[1]{}







\usepackage{homework,stmaryrd,wasysym,url,upgreek,subfigure}
\usepackage[margin=1cm]{caption}
\usepackage{tikz-dependency}
\DeclareMathAlphabet{\mathkw}{OT1}{cmss}{bx}{n}

\begin{document}
\setmainfont{Times New Roman}

\author{Jonathan Sterling}

\title{Phrase Projectivity in Antigone}
\maketitle

\ignore{
\begin{hscode}\SaveRestoreHook
\column{B}{@{}>{\hspre}l<{\hspost}@{}}%
\column{3}{@{}>{\hspre}l<{\hspost}@{}}%
\column{E}{@{}>{\hspre}l<{\hspost}@{}}%
\>[3]{}\Comment{\{-\#\enskip LANGUAGE StandaloneDeriving \enskip\#-\}}{}\<[E]%
\ColumnHook
\end{hscode}\resethooks
\begin{hscode}\SaveRestoreHook
\column{B}{@{}>{\hspre}l<{\hspost}@{}}%
\column{3}{@{}>{\hspre}l<{\hspost}@{}}%
\column{E}{@{}>{\hspre}l<{\hspost}@{}}%
\>[3]{}\mathkw{module}\;\mathsf{Analyze}\;\mathkw{where}{}\<[E]%
\\
\>[3]{}\mathkw{import}\;\mathsf{\mathsf{Control}.Applicative}{}\<[E]%
\\
\>[3]{}\mathkw{import}\;\mathsf{\mathsf{Control}.Arrow}\;((\VarSym{\&\&\&})){}\<[E]%
\\
\>[3]{}\mathkw{import}\;\mathsf{\mathsf{Data}.Foldable}{}\<[E]%
\\
\>[3]{}\mathkw{import}\;\mathsf{\mathsf{Data}.Monoid}{}\<[E]%
\\
\>[3]{}\mathkw{import}\;\mathsf{\mathsf{Data}.Tree}{}\<[E]%
\\
\>[3]{}\mathkw{import}\;\mathsf{\mathsf{Data}.List}\;(\Varid{genericLength}){}\<[E]%
\\
\>[3]{}\mathkw{import}\;\mathsf{Prelude}\;\Varid{hiding}\;(\FN{maximum},\FN{minimum},\FN{foldl},\cdot \notin\cdot ){}\<[E]%
\ColumnHook
\end{hscode}\resethooks
}

\section{Introduction}
\label{sec:introduction}

\subsection{Dependency Trees and Projectivity}
\label{sec:introduction.dependency}

A dependency tree encodes the head-dependent relation for a string of words,
where arcs are drawn from heads to their dependents. We consider a phrase
\emph{projective} when these arcs do not cross each other, and
\emph{discontinuous} inasmuch as any of the arcs do cross.
Figure~\ref{fig:dependency-trees} illustrates the various kinds of projectivity
violations that may occur.

\begin{figure}[h!]
\centering
\subfigure[``Full of plentiful supplies'' (Xenophon, \emph{Anabasis} 3.5.1).]{
  \begin{dependency}[theme = simple]
      \begin{deptext}[column sep=1em]
          μεστῇ \& πολλῶν \& ἀγαθῶν\\
      \end{deptext}
      \depedge{3}{2}{ADJ}
      \depedge{1}{3}{GEN}
  \end{dependency}
}
\hspace{6pt}
\subfigure[``Full of many good things'' (Plato, \emph{Laws} 906a).]{
  \begin{dependency}[theme = simple]
      \begin{deptext}[column sep=1em]
          πολλῶν \& μεστὸν \& ἀγαθῶν\\
      \end{deptext}
      \depedge[edge start x offset=-6pt, arc angle=20]{3}{1}{ADJ}
      \depedge{2}{3}{GEN}
  \end{dependency}
}
\subfigure[``And he stood over the rooftops, gaped in a circle with murderous
spears around the seven-gated mouth, and left.'' (Sophocles, \emph{Antigone}
117--120).]{
  \begin{dependency}[theme = simple]
      \begin{deptext}[column sep=0.1em]
        στὰς \& δ' \& ὑπὲρ \& μελάθρων \& φονώσαισιν \& ἀμφιχανὼν \& κύκλῳ \&
λόγχαις \& ἑπτάπυλον \& στόμα \& ἔβα \\
      \end{deptext}
      \depedge{3}{4}{OBJ}
      \depedge{1}{3}{ADV}
      \depedge{6}{7}{ADV}
      \depedge{8}{5}{ADJ}
      \depedge{6}{8}{ADV}
      \depedge{10}{9}{ADJ}
      \depedge{6}{10}{OBJ}
      \depedge{11}{6}{ADV}
      \depedge{11}{1}{ADV}
      \depedge{11}{2}{CONJ}
  \end{dependency}
}

\caption{A dependency path wrapping around itself is a projectivity violation,
as in (b); interlacing adjacent phrases also violate projectivity, as in
(c). Examples (a--b) drawn from Devine~\&~Stephens.}
\label{fig:dependency-trees}
\end{figure}

\begin{hscode}\SaveRestoreHook
\column{B}{@{}>{\hspre}l<{\hspost}@{}}%
\column{3}{@{}>{\hspre}l<{\hspost}@{}}%
\column{E}{@{}>{\hspre}l<{\hspost}@{}}%
\>[3]{}\mathkw{data}\;\mathsf{Tree}\;\alpha\mathrel{=}\alpha\curvearrowright[\mskip1.5mu \mathsf{Tree}\;\alpha\mskip1.5mu]{}\<[E]%
\ColumnHook
\end{hscode}\resethooks
\begin{hscode}\SaveRestoreHook
\column{B}{@{}>{\hspre}l<{\hspost}@{}}%
\column{3}{@{}>{\hspre}l<{\hspost}@{}}%
\column{E}{@{}>{\hspre}l<{\hspost}@{}}%
\>[3]{}\FN{getLabel}\ConSym{::}\mathsf{Tree}\;\alpha\to \alpha{}\<[E]%
\\
\>[3]{}\FN{getLabel}\;(\Varid{l}\curvearrowright\anonymous )\mathrel{=}\Varid{l}{}\<[E]%
\ColumnHook
\end{hscode}\resethooks
\begin{hscode}\SaveRestoreHook
\column{B}{@{}>{\hspre}l<{\hspost}@{}}%
\column{3}{@{}>{\hspre}l<{\hspost}@{}}%
\column{21}{@{}>{\hspre}l<{\hspost}@{}}%
\column{E}{@{}>{\hspre}l<{\hspost}@{}}%
\>[3]{}\mathkw{type}\;\mathsf{Edge}\;\alpha{}\<[21]%
\>[21]{}\mathrel{=}(\alpha,\alpha){}\<[E]%
\\
\>[3]{}\mathkw{data}\;\mathsf{Range}\;\alpha{}\<[21]%
\>[21]{}\mathrel{=}\alpha\leftrightarrow\alpha\mid \FN{R_\emptyset}{}\<[E]%
\\
\>[3]{}\mathkw{data}\;\mathsf{RangeState}\;\alpha\mathrel{=}\mathsf{Integer}\lhd\mathsf{Range}\;\alpha{}\<[E]%
\ColumnHook
\end{hscode}\resethooks
\begin{hscode}\SaveRestoreHook
\column{B}{@{}>{\hspre}l<{\hspost}@{}}%
\column{3}{@{}>{\hspre}l<{\hspost}@{}}%
\column{E}{@{}>{\hspre}l<{\hspost}@{}}%
\>[3]{}\FN{getRange}\ConSym{::}\mathsf{RangeState}\;\alpha\to \mathsf{Range}\;\alpha{}\<[E]%
\\
\>[3]{}\FN{getRange}\;(\anonymous \lhd\Varid{r})\mathrel{=}\Varid{r}{}\<[E]%
\ColumnHook
\end{hscode}\resethooks
\begin{hscode}\SaveRestoreHook
\column{B}{@{}>{\hspre}l<{\hspost}@{}}%
\column{3}{@{}>{\hspre}l<{\hspost}@{}}%
\column{E}{@{}>{\hspre}l<{\hspost}@{}}%
\>[3]{}\FN{getViolations}\ConSym{::}\mathsf{RangeState}\;\alpha\to \mathsf{Integer}{}\<[E]%
\\
\>[3]{}\FN{getViolations}\;(\Varid{vs}\lhd\anonymous )\mathrel{=}\Varid{vs}{}\<[E]%
\ColumnHook
\end{hscode}\resethooks
\ignore{
\begin{hscode}\SaveRestoreHook
\column{B}{@{}>{\hspre}l<{\hspost}@{}}%
\column{3}{@{}>{\hspre}l<{\hspost}@{}}%
\column{E}{@{}>{\hspre}l<{\hspost}@{}}%
\>[3]{}\mathkw{deriving}\;\mathkw{instance}\;\mathsf{Show}\;\alpha\Rightarrow \mathsf{Show}\;(\mathsf{Range}\;\alpha){}\<[E]%
\\
\>[3]{}\mathkw{deriving}\;\mathkw{instance}\;\mathsf{Show}\;\alpha\Rightarrow \mathsf{Show}\;(\mathsf{RangeState}\;\alpha){}\<[E]%
\ColumnHook
\end{hscode}\resethooks
}

\noindent
A \ensuremath{\mathsf{Monoid}} is an algebraic structure which has a zero \ensuremath{\mathcal{E}} and a binary
operation \ensuremath{\cdot \oplus\cdot }, and which satisfies some laws:
\begin{hscode}\SaveRestoreHook
\column{B}{@{}>{\hspre}l<{\hspost}@{}}%
\column{3}{@{}>{\hspre}l<{\hspost}@{}}%
\column{5}{@{}>{\hspre}l<{\hspost}@{}}%
\column{19}{@{}>{\hspre}l<{\hspost}@{}}%
\column{E}{@{}>{\hspre}l<{\hspost}@{}}%
\>[3]{}\mathkw{class}\;\mathsf{Monoid}\;\alpha\;\mathkw{where}{}\<[E]%
\\
\>[3]{}\hsindent{2}{}\<[5]%
\>[5]{}\mathcal{E}\ConSym{::}\alpha{}\<[E]%
\\
\>[3]{}\hsindent{2}{}\<[5]%
\>[5]{}\cdot \oplus\cdot \ConSym{::}\alpha\to \alpha\to \alpha{}\<[E]%
\\[\blanklineskip]%
\>[3]{}\hsindent{2}{}\<[5]%
\>[5]{}\FN{associativity}\ConSym{::}\Varid{l}\oplus(\Varid{c}\oplus\Varid{r})\equiv (\Varid{l}\oplus\Varid{c})\oplus\Varid{r}{}\<[E]%
\\
\>[3]{}\hsindent{2}{}\<[5]%
\>[5]{}\FN{leftIdentity}{}\<[19]%
\>[19]{}\ConSym{::}\Varid{l}\oplus\mathcal{E}\equiv \Varid{l}{}\<[E]%
\\
\>[3]{}\hsindent{2}{}\<[5]%
\>[5]{}\FN{rightIdentity}\ConSym{::}\mathcal{E}\oplus\Varid{l}\equiv \Varid{l}{}\<[E]%
\ColumnHook
\end{hscode}\resethooks
\begin{hscode}\SaveRestoreHook
\column{B}{@{}>{\hspre}l<{\hspost}@{}}%
\column{3}{@{}>{\hspre}l<{\hspost}@{}}%
\column{5}{@{}>{\hspre}l<{\hspost}@{}}%
\column{E}{@{}>{\hspre}l<{\hspost}@{}}%
\>[3]{}\mathkw{instance}\;\mathsf{Ord}\;\alpha\Rightarrow \mathsf{Monoid}\;(\mathsf{Range}\;\alpha)\;\mathkw{where}{}\<[E]%
\\
\>[3]{}\hsindent{2}{}\<[5]%
\>[5]{}\mathcal{E}\mathrel{=}\FN{R_\emptyset}{}\<[E]%
\\
\>[3]{}\hsindent{2}{}\<[5]%
\>[5]{}(\Varid{x}\leftrightarrow\Varid{y})\oplus(\Varid{u}\leftrightarrow\Varid{v})\mathrel{=}\FN{rangeFrom}\;[\mskip1.5mu \Varid{x},\Varid{y},\Varid{u},\Varid{v}\mskip1.5mu]{}\<[E]%
\\
\>[3]{}\hsindent{2}{}\<[5]%
\>[5]{}\FN{R_\emptyset}\oplus\Varid{xy}\mathrel{=}\Varid{xy}{}\<[E]%
\\
\>[3]{}\hsindent{2}{}\<[5]%
\>[5]{}\Varid{xy}\oplus\FN{R_\emptyset}\mathrel{=}\Varid{xy}{}\<[E]%
\ColumnHook
\end{hscode}\resethooks
\begin{hscode}\SaveRestoreHook
\column{B}{@{}>{\hspre}l<{\hspost}@{}}%
\column{3}{@{}>{\hspre}l<{\hspost}@{}}%
\column{5}{@{}>{\hspre}l<{\hspost}@{}}%
\column{7}{@{}>{\hspre}l<{\hspost}@{}}%
\column{18}{@{}>{\hspre}l<{\hspost}@{}}%
\column{E}{@{}>{\hspre}l<{\hspost}@{}}%
\>[3]{}\mathkw{instance}\;(\mathsf{Num}\;\alpha,\mathsf{Ord}\;\alpha)\Rightarrow \mathsf{Monoid}\;(\mathsf{RangeState}\;\alpha)\;\mathkw{where}{}\<[E]%
\\
\>[3]{}\hsindent{2}{}\<[5]%
\>[5]{}\mathcal{E}\mathrel{=}\Numeral{0}\lhd\mathcal{E}{}\<[E]%
\\
\>[3]{}\hsindent{2}{}\<[5]%
\>[5]{}(\Varid{i}\lhd\Varid{xy})\oplus(\Varid{j}\lhd\Varid{uv})\mathrel{=}\Varid{count}\lhd(\Varid{xy}\oplus\Varid{uv})\;\mathkw{where}{}\<[E]%
\\
\>[5]{}\hsindent{2}{}\<[7]%
\>[7]{}\Varid{count}\mathrel{=}\mathkw{if}\;\FN{rangesIntersect}\;\Varid{xy}\;\Varid{uv}{}\<[E]%
\\
\>[7]{}\hsindent{11}{}\<[18]%
\>[18]{}\mathkw{then}\;\Varid{i}\VarSym{+}\Varid{j}\VarSym{+}\Numeral{1}{}\<[E]%
\\
\>[7]{}\hsindent{11}{}\<[18]%
\>[18]{}\mathkw{else}\;\Varid{i}\VarSym{+}\Varid{j}{}\<[E]%
\ColumnHook
\end{hscode}\resethooks
\begin{hscode}\SaveRestoreHook
\column{B}{@{}>{\hspre}l<{\hspost}@{}}%
\column{3}{@{}>{\hspre}l<{\hspost}@{}}%
\column{5}{@{}>{\hspre}l<{\hspost}@{}}%
\column{29}{@{}>{\hspre}l<{\hspost}@{}}%
\column{39}{@{}>{\hspre}l<{\hspost}@{}}%
\column{E}{@{}>{\hspre}l<{\hspost}@{}}%
\>[3]{}\FN{rangesIntersect}\ConSym{::}\mathsf{Ord}\;\alpha\Rightarrow \mathsf{Range}\;\alpha\to \mathsf{Range}\;\alpha\to \mathsf{Bool}{}\<[E]%
\\
\>[3]{}\FN{rangesIntersect}\;(\Varid{x}\leftrightarrow\Varid{y})\;(\Varid{u}\leftrightarrow\Varid{v})\mathrel{=}{}\<[E]%
\\
\>[3]{}\hsindent{2}{}\<[5]%
\>[5]{}\neg \;((\Varid{x}\VarSym{<}\Varid{u}\mathrel{\wedge}\Varid{y}\VarSym{<}\Varid{u})\mathrel{\vee}(\Varid{u}\VarSym{<}\Varid{v}\mathrel{\wedge}\Varid{v}\VarSym{<}\Varid{x})){}\<[E]%
\\
\>[3]{}\FN{rangesIntersect}\;\anonymous \;{}\<[29]%
\>[29]{}\anonymous {}\<[39]%
\>[39]{}\mathrel{=}\mathsf{False}{}\<[E]%
\ColumnHook
\end{hscode}\resethooks
\begin{hscode}\SaveRestoreHook
\column{B}{@{}>{\hspre}l<{\hspost}@{}}%
\column{3}{@{}>{\hspre}l<{\hspost}@{}}%
\column{E}{@{}>{\hspre}l<{\hspost}@{}}%
\>[3]{}\FN{rangeFrom}\ConSym{::}(\mathsf{Foldable}\;\phi,\mathsf{Ord}\;\alpha)\Rightarrow \phi\;\alpha\to \mathsf{Range}\;\alpha{}\<[E]%
\\
\>[3]{}\FN{rangeFrom}\;\Varid{xs}\mathrel{=}\FN{minimum}\;\Varid{xs}\leftrightarrow\FN{maximum}\;\Varid{xs}{}\<[E]%
\ColumnHook
\end{hscode}\resethooks
\begin{hscode}\SaveRestoreHook
\column{B}{@{}>{\hspre}l<{\hspost}@{}}%
\column{3}{@{}>{\hspre}l<{\hspost}@{}}%
\column{5}{@{}>{\hspre}l<{\hspost}@{}}%
\column{23}{@{}>{\hspre}l<{\hspost}@{}}%
\column{E}{@{}>{\hspre}l<{\hspost}@{}}%
\>[3]{}\FN{analyzePath}\ConSym{::}(\mathsf{Num}\;\alpha,\mathsf{Ord}\;\alpha)\Rightarrow [\mskip1.5mu \alpha\mskip1.5mu]\to \mathsf{RangeState}\;\alpha{}\<[E]%
\\
\>[3]{}\FN{analyzePath}\;\Varid{path}\mathrel{=}\FN{foldl}\;\Varid{op}\;\mathcal{E}\;(\Varid{reverse}\;\Varid{path})\;\mathkw{where}{}\<[E]%
\\
\>[3]{}\hsindent{2}{}\<[5]%
\>[5]{}\Varid{op}\;(\Varid{c}\lhd\Varid{r})\;\Varid{i}\mathrel{=}\mathkw{if}\;\FN{inRange}\;\Varid{r}\;\Varid{i}{}\<[E]%
\\
\>[5]{}\hsindent{18}{}\<[23]%
\>[23]{}\mathkw{then}\;(\Varid{c}\VarSym{+}\Numeral{1})\lhd\Varid{r}{}\<[E]%
\\
\>[5]{}\hsindent{18}{}\<[23]%
\>[23]{}\mathkw{else}\;\Varid{c}\lhd(\FN{extend}\;\Varid{r}\;\Varid{i}){}\<[E]%
\ColumnHook
\end{hscode}\resethooks
\begin{hscode}\SaveRestoreHook
\column{B}{@{}>{\hspre}l<{\hspost}@{}}%
\column{3}{@{}>{\hspre}l<{\hspost}@{}}%
\column{5}{@{}>{\hspre}l<{\hspost}@{}}%
\column{7}{@{}>{\hspre}l<{\hspost}@{}}%
\column{9}{@{}>{\hspre}l<{\hspost}@{}}%
\column{18}{@{}>{\hspre}l<{\hspost}@{}}%
\column{24}{@{}>{\hspre}l<{\hspost}@{}}%
\column{E}{@{}>{\hspre}l<{\hspost}@{}}%
\>[3]{}\FN{analyzeTree}\ConSym{::}(\mathsf{Num}\;\alpha,\mathsf{Ord}\;\alpha)\Rightarrow \mathsf{Tree}\;\alpha\to \mathsf{Tree}\;(\mathsf{RangeState}\;\alpha){}\<[E]%
\\
\>[3]{}\FN{analyzeTree}\;\Varid{tree}\mathrel{=}{}\<[E]%
\\
\>[3]{}\hsindent{2}{}\<[5]%
\>[5]{}\mathkw{case}\;\FN{treeOrPath}\;\Varid{tree}\;\mathkw{of}{}\<[E]%
\\
\>[5]{}\hsindent{2}{}\<[7]%
\>[7]{}\mathsf{Left}\;(\Varid{i}\curvearrowright\Varid{ts})\to \Varid{c'}\lhd\FN{extend}\;\Varid{r}\;\Varid{i}\curvearrowright\Varid{children}\;\mathkw{where}{}\<[E]%
\\
\>[7]{}\hsindent{2}{}\<[9]%
\>[9]{}\Varid{children}\mathrel{=}\FN{analyzeTree}\mathop{\langle\$\rangle}\Varid{ts}{}\<[E]%
\\
\>[7]{}\hsindent{2}{}\<[9]%
\>[9]{}\Varid{c}\lhd\Varid{r}{}\<[18]%
\>[18]{}\mathrel{=}\FN{fold}\;(\FN{getLabel}\mathop{\langle\$\rangle}\Varid{children}){}\<[E]%
\\
\>[7]{}\hsindent{2}{}\<[9]%
\>[9]{}\Varid{c'}\mathrel{=}\Varid{c}\VarSym{+}(\Varid{genericLength}\;(\Varid{filter}\;(\lambda \Varid{r'}\to \FN{inRange}\;\Varid{r'}\;\Varid{i})\;(\FN{getRange}\circ\FN{getLabel}\mathop{\langle\$\rangle}\Varid{children}))){}\<[E]%
\\
\>[5]{}\hsindent{2}{}\<[7]%
\>[7]{}\mathsf{Right}\;\Varid{path}{}\<[24]%
\>[24]{}\to \FN{analyzePath}\;\Varid{path}\curvearrowright[\mskip1.5mu \mskip1.5mu]{}\<[E]%
\ColumnHook
\end{hscode}\resethooks
\begin{hscode}\SaveRestoreHook
\column{B}{@{}>{\hspre}l<{\hspost}@{}}%
\column{3}{@{}>{\hspre}l<{\hspost}@{}}%
\column{27}{@{}>{\hspre}l<{\hspost}@{}}%
\column{E}{@{}>{\hspre}l<{\hspost}@{}}%
\>[3]{}\FN{treeOrPath}\ConSym{::}\mathsf{Tree}\;\alpha\to \mathsf{Either}\;(\mathsf{Tree}\;\alpha)\;[\mskip1.5mu \alpha\mskip1.5mu]{}\<[E]%
\\
\>[3]{}\FN{treeOrPath}\;(\Varid{i}\curvearrowright[\mskip1.5mu \mskip1.5mu]){}\<[27]%
\>[27]{}\mathrel{=}\mathsf{Right}\;[\mskip1.5mu \Varid{i}\mskip1.5mu]{}\<[E]%
\\
\>[3]{}\FN{treeOrPath}\;(\Varid{i}\curvearrowright[\mskip1.5mu \Varid{x}\mskip1.5mu])\mathrel{=}(\Varid{i}\mathbin{:})\mathop{\langle\$\rangle}\FN{treeOrPath}\;\Varid{x}{}\<[E]%
\\
\>[3]{}\FN{treeOrPath}\;\Varid{t}{}\<[27]%
\>[27]{}\mathrel{=}\mathsf{Left}\;\Varid{t}{}\<[E]%
\ColumnHook
\end{hscode}\resethooks
\begin{hscode}\SaveRestoreHook
\column{B}{@{}>{\hspre}l<{\hspost}@{}}%
\column{3}{@{}>{\hspre}l<{\hspost}@{}}%
\column{E}{@{}>{\hspre}l<{\hspost}@{}}%
\>[3]{}\FN{extend}\ConSym{::}\mathsf{Ord}\;\alpha\Rightarrow \mathsf{Range}\;\alpha\to \alpha\to \mathsf{Range}\;\alpha{}\<[E]%
\\
\>[3]{}\FN{extend}\;(\Varid{x}\leftrightarrow\Varid{y})\;\Varid{z}\mathrel{=}\FN{rangeFrom}\;[\mskip1.5mu \Varid{x},\Varid{y},\Varid{z}\mskip1.5mu]{}\<[E]%
\\
\>[3]{}\FN{extend}\;\FN{R_\emptyset}\;\Varid{z}\mathrel{=}\Varid{z}\leftrightarrow\Varid{z}{}\<[E]%
\ColumnHook
\end{hscode}\resethooks
\begin{hscode}\SaveRestoreHook
\column{B}{@{}>{\hspre}l<{\hspost}@{}}%
\column{3}{@{}>{\hspre}l<{\hspost}@{}}%
\column{E}{@{}>{\hspre}l<{\hspost}@{}}%
\>[3]{}\FN{inRange}\ConSym{::}\mathsf{Ord}\;\alpha\Rightarrow \mathsf{Range}\;\alpha\to \alpha\to \mathsf{Bool}{}\<[E]%
\\
\>[3]{}\FN{inRange}\;(\Varid{x}\leftrightarrow\Varid{y})\;\Varid{z}\mathrel{=}\Varid{z}\VarSym{>}\Varid{x}\mathrel{\wedge}\Varid{z}\VarSym{<}\Varid{y}{}\<[E]%
\\
\>[3]{}\FN{inRange}\;\FN{R_\emptyset}\;\anonymous \mathrel{=}\mathsf{False}{}\<[E]%
\ColumnHook
\end{hscode}\resethooks
\begin{hscode}\SaveRestoreHook
\column{B}{@{}>{\hspre}l<{\hspost}@{}}%
\column{3}{@{}>{\hspre}l<{\hspost}@{}}%
\column{4}{@{}>{\hspre}l<{\hspost}@{}}%
\column{E}{@{}>{\hspre}l<{\hspost}@{}}%
\>[3]{}\FN{maximalPoint}\ConSym{::}\mathsf{Eq}\;\alpha\Rightarrow [\mskip1.5mu \mathsf{Edge}\;\alpha\mskip1.5mu]\to \mathsf{Maybe}\;\alpha{}\<[E]%
\\
\>[3]{}\FN{maximalPoint}\;\Varid{es}\mathrel{=}{}\<[E]%
\\
\>[3]{}\hsindent{1}{}\<[4]%
\>[4]{}\FN{find}\;(\lambda \Varid{x}\to \Varid{x}\notin\FN{snd}\mathop{\langle\$\rangle}\Varid{es})\;(\FN{fst}\mathop{\langle\$\rangle}\Varid{es}){}\<[E]%
\ColumnHook
\end{hscode}\resethooks
\end{document}


